\documentclass[border=5pt]{standalone}
%
\usepackage{basicsty}
%
\begin{document} 

%%%%%%%%%%%%%%%%%%%%%%%%%%%%%%%%%%% e_scanner.pdf
% \newcommand{\token}[2]{$\langle$ \texttt{\bfseries #1}, \texttt{#2} $\rangle$}
% \newcommand{\self}[1]{$\langle$ \texttt{\bfseries #1} $\rangle$}
% \begin{tikzpicture}
%     \node[rect1] (scanner) {Scanner};
%     \coordinate[left = 2cm of scanner] (lp);
%     \coordinate[right = 2cm of scanner] (rp);
    
%     \coordinate (lrt) at ($ (lp) + (-1.5cm, 0cm) $);
%     \coordinate (rrt) at ($ (rp) + (+2cm, 0cm) $);

%     \node at (lrt) {
%       \begin{minipage}{2.5cm}
% \begin{lstlisting}[numbers=none]
% if [1 = a]
% {
%     res = "ok";
% }
% \end{lstlisting}
%       \end{minipage}
%     }
%     ;
%     \node at (rrt) {
%       \begin{minipage}{4cm}
%           % \begin{center}
%           \self{if}
%           \self{[}
%           \token{INT}{$1$}
%           \self{=}

%           \token{ID}{"a"}
%           \self{]}
%           $\langle \{\rangle$

%           \token{ID}{"res"}
%           \self{:}

%           \token{STR}{"ok"}
%           \self{;}
%           $\langle \}\rangle$
%           % \end{center}
%       \end{minipage}
%     }
%     ;

%     % 绘制中间连线
%     \draw[arrow1]
%         (lp)
%         -- node[above] {字符流} 
%         (scanner)
%         ;
%     \draw[arrow1]
%         (scanner)
%         -- node[above]{token流} 
%         (rp)
%         ;
% \end{tikzpicture} 

%%%%%%%%%%%%%%%%%%%%%%%%%%%%%%%%%%% e_parser.pdf
% \newcommand{\token}[2]{$\langle$ \texttt{\bfseries #1}, \texttt{#2} $\rangle$}
% \newcommand{\self}[1]{$\langle$ \texttt{\bfseries #1} $\rangle$}
% \begin{tikzpicture}[level distance=1.5cm,
%       level 1/.style={sibling distance=2cm, level distance=1.3cm},
%       level 2/.style={sibling distance=1.3cm, level distance=1cm}]
%     \node[rect1] (parser) {Parser};
%     \coordinate[left = 2cm of parser] (lp);
%     \coordinate[right = 2cm of parser] (rp);
    
%     \coordinate (lrt) at ($ (lp) + (-1.5cm, 0cm) $);
%     \coordinate (rrt) at ($ (rp) + (+1.7cm, 1.5cm) $);

%     \node at (lrt) {
%       \begin{minipage}{4cm}
%           % \begin{center}
%           \self{if}
%           \self{[}
%           \token{INT}{$1$}
%           \self{=}

%           \token{ID}{"a"}
%           \self{]}
%           $\langle \{\rangle$

%           \token{ID}{"res"}
%           \self{:}

%           \token{STR}{"ok"}
%           \self{;}
%           $\langle \}\rangle$
%           % \end{center}
%       \end{minipage}
%     }
%     ;
%     \node at (rrt) {if\_statement}
%     child{node{$=$} child{node{1}} child{node{a}}}
%     child{node{block} child{node {$:$} child{node{res}} child{node{\texttt{"ok"}}}}}
%     ;

%     % 绘制中间连线
%     \draw[arrow1]
%         (lp)
%         -- node[above] {token流} 
%         (parser)
%         ;
%     \draw[arrow1]
%         (parser)
%         -- node[above]{初步构建} node[below]{的语法树} 
%         (rp)
%         ;
% \end{tikzpicture} 




%%%%%%%%%%%%%%%%%%%%%%%%%%%%%%% semantic.pdf
% \newcommand{\token}[1]{\texttt{\bfseries #1}}
% \begin{tikzpicture}[level distance=1.5cm,
%     level 1/.style={sibling distance=2cm, level distance=1.3cm},
%     level 2/.style={sibling distance=1.3cm, level distance=1cm}]
    
%     \node [rect1] (semantic) {语义分析子程序};
%     \coordinate [left = 2cm of semantic] (lp);
%     \coordinate [right = 2cm of semantic] (rp);

%     \coordinate (lrt) at ($ (lp) + (-1.5cm, 1.3cm) $);
%     \coordinate (rrt) at ($ (rp) + (+1.5cm, 1.3cm) $);
    
%     \node at (lrt) {$+$}
%     child{node{1.5}}
%     child{node{2}}
%     ;
%     \node at (rrt) {$+$}
%     child{node{1.5}}
%     child{node{\token{inttofloat}} child{node{2}}}
%     ;
%     % 绘制中间连线
%     \draw[arrow1]
%         (lp)
%         -- node[above] {初步构建}  node[below] {的语法树}
%         (semantic);
%     \draw[arrow1]
%         (semantic) 
%         -- node[above]{修正后}   node[below] {的语法树}
%         (rp);
% \end{tikzpicture} 


%%%%%%%%%%%%%%%%%%%%%%%%%%%%%%% semantic1.pdf
% \newcommand{\token}[1]{\texttt{\bfseries #1}}
% \begin{tikzpicture}[level distance=1.5cm,
%     level 1/.style={sibling distance=2cm, level distance=1.3cm},
%     level 2/.style={sibling distance=1.3cm, level distance=1cm}]
    
%     \coordinate (lp);
%     \coordinate [right = of lp] (rp);
%     \coordinate (lrt) at ($ (lp) + (-1.7cm, 1.3cm) $);
%     \coordinate (rrt) at ($ (rp) + (+1.7cm, 1.3cm) $);
%     \node at (lrt) {$>$}
%     child{node{$>$} child{node{3}} child{node{2}}}
%     child{node{1}}
%     ; 
%     \node at (rrt) {$\&\&$}
%     child{node{$>$} child{node{3}} child{node{2}}}
%     child{node{$>$} child{node{2}} child{node{1}}}
%     ;

%     \draw [arrow1] (lp) -- node[above]{修正} (rp);
% \end{tikzpicture} 


%%%%%%%%%%%%%%%%%%%%%%%%%%%%%%% semantic3.pdf
% \newcommand{\token}[1]{\texttt{\bfseries #1}}
% \begin{tikzpicture}[level distance=1.5cm,
%     level 1/.style={sibling distance=2cm, level distance=1.3cm},
%     level 2/.style={sibling distance=1.3cm, level distance=1cm}]
    
%     \coordinate (lp);
%     \coordinate [right = of lp] (rp);
%     \coordinate (lrt) at ($ (lp) + (-1.7cm, 0.65cm) $);
%     \coordinate (rrt) at ($ (rp) + (+2cm, 0cm) $);
%     \node at (lrt) {\token{repeat\_statement}}
%     child{node{1.5}}
%     child{node{\token{block}}}
%     ; 
%     \node at (rrt) {{\parbox[c]{3cm}{The expression in the repeat statement is not a integer type.}}}
%     ;

%     \draw [arrow1] (lp) -- node[above]{报错} (rp);
% \end{tikzpicture} 


%%%%%%%%%%%%%%%%%%%%%%%%%%%%%%% semantic4.pdf
% \newcommand{\token}[1]{\texttt{\bfseries #1}}
% \begin{tikzpicture}[level distance=1.5cm,
%     level 1/.style={sibling distance=2cm, level distance=1.3cm},
%     level 2/.style={sibling distance=1.3cm, level distance=1cm}]
    
%     \coordinate (lp);
%     \coordinate [right = of lp] (rp);
%     \coordinate (lrt) at ($ (lp) + (-2cm, 1.3cm) $);
%     \coordinate (rrt) at ($ (rp) + (+2cm, 0cm) $);
%     \node at (lrt) {\token{if\_statement}}
%     child{node{$-$} child{node{3}} child{node{2}}}
%     child{node{\token{block}}}
%     ; 
%     \node at (rrt) {{\parbox[c]{3cm}{The expression in the if statement is not a boolean type.}}}
%     ;

%     \draw [arrow1] (lp) -- node[above]{报错} (rp);
% \end{tikzpicture} 

%%%%%%%%%%%%%%%%%%%%%%%%%%%%%%% semantic5.pdf
% \newcommand{\token}[1]{\texttt{\bfseries #1}}
% \begin{tikzpicture}[level distance=1.5cm,
%     level 1/.style={sibling distance=2.5cm, level distance=1.3cm},
%     level 2/.style={sibling distance=1.3cm, level distance=1cm}]
    
%     \coordinate (lp);
%     \coordinate [right = of lp] (rp);
%     \coordinate (lrt) at ($ (lp) + (-2.5cm, 1.3cm) $);
%     \coordinate (rrt) at ($ (rp) + (+2cm, 0cm) $);
%     \node at (lrt) {statements}
%     child{node{declare} child{node{\token{integer}}} child{node{\token{a}}}}
%     child{node{declare} child{node{\token{float}}} child{node{\token{a}}}}
%     ; 
%     \node at (rrt) {{\parbox[c]{3cm}{Duplicate variable name \texttt{"a"}.}}}
%     ;

%     \draw [arrow1] (lp) -- node[above]{报错} (rp);
% \end{tikzpicture} 


%%%%%%%%%%%%%%%%%%%%%%%%%%%%%%% e_generator.pdf
% \newcommand{\token}[1]{\texttt{\bfseries #1}}
% \begin{tikzpicture}[level distance=1.5cm,
%     level 1/.style={sibling distance=2cm, level distance=1.3cm},
%     level 2/.style={sibling distance=1.3cm, level distance=1cm}]
%     \node [rect1] (generator) {代码生成子程序};
%     \coordinate [left = 2cm of generator] (lp);
%     \coordinate [right = 2cm of generator] (rp);
%     \coordinate (lrt) at ($ (lp) + (-2.3cm, 1.3cm) $);
%     \coordinate (rrt) at ($ (rp) + (+2.3cm, 0cm) $);
    
%     \node at (lrt) {$+$}
%     child{node{1.5}}
%     child{node{\token{inttofloat}} child{node{2}}}
%     ;

%     \node at (rrt) {
%         \begin{minipage}{4cm}
% Bytecode:
% \vspace{-.1cm}
% \begin{lstlisting}[numbers=none]
% 0000: push_float         0000
% 0003: push_byte          02
% 0005: cast_int_to_float 
% 0006: add_float         
% \end{lstlisting}
% Constant Pool:
% \vspace{-.1cm}
% \begin{lstlisting}[numbers=none]
% 0000: 1.5
% \end{lstlisting}
%         \end{minipage}
%     }
%     ;
%     % 绘制中间连线
%     \draw[arrow1]
%         (lp)
%         -- node[above] {修正后} node[below] {的语法树}
%         (generator);
%     \draw[arrow1]
%         (generator) 
%         -- node[above]{字节码} 
%         (rp);
% \end{tikzpicture} 


%%%%%%%%%%%%%%%%%%%%%%%%%%%%%%% generator2.pdf
% \newcommand{\token}[1]{\texttt{\bfseries #1}}
% \begin{tikzpicture}[level distance=1.5cm,
%     level 1/.style={sibling distance=2cm, level distance=1.3cm},
%     level 2/.style={sibling distance=1.2cm, level distance=1cm}]
    
%     \coordinate (lp);
%     \coordinate [right = of lp] (rp);
%     \coordinate (lrt) at ($ (lp) + (-2cm, 1.7cm) $);
%     \coordinate (rrt) at ($ (rp) + (+2.7cm, 0cm) $);
%     \node at (lrt) {statements}
%     child{node{declare} child{node{\token{integer}}} child{node{\token{a}}}}
%     child{node{expression} child{node{$:$} child{node{\token{a}}} child{node{1}}}}
%     ;
%     \node at (rrt) {
%         \begin{minipage}{4cm}
% Bytecode:
% \vspace{-.1cm}
% \begin{lstlisting}[numbers=none]
% 0000: push_byte          01
% 0002: duplicate         
% 0003: pop_static_int     0000
% 0006: pop
% \end{lstlisting}
% Static variables:
% \vspace{-.1cm}
% \begin{lstlisting}[numbers=none]
% 0000: integer a
% \end{lstlisting}
%         \end{minipage}
%     }
%     ;

%     \draw [arrow1] (lp) 
%     -- node [above] {代码} node [below] {生成}
%     (rp);
% \end{tikzpicture} 

%%%%%%%%%%%%%%%%%%%%%%%%%%%%%%% generator3.pdf
% \newcommand{\token}[1]{\texttt{\bfseries #1}}
% \begin{tikzpicture}[level distance=1.5cm,
%     level 1/.style={sibling distance=3cm, level distance=1.3cm},
%     level 2/.style={sibling distance=1.2cm, level distance=1cm}]
    
%     \coordinate (lp);
%     \coordinate [right = of lp] (rp);
%     \coordinate (lrt) at ($ (lp) + (-2.5cm, 1.7cm) $);
%     \coordinate (rrt) at ($ (rp) + (+2.7cm, 0cm) $);
%     \node at (lrt) {statements}
%     child{
%       node{function} 
%       child{node{\token{integer}}} 
%       child{node{\token{max}}}
%       child{node{params} child{node{略}}}
%     }
%     child{node{expression} child{node{\token{max}} child{node{1}} child{node{2}}}}
%     ;
%     \node at (rrt) {
%         \begin{minipage}{4cm}
% Bytecode:
% \vspace{-.1cm}
% \begin{lstlisting}[numbers=none]
% 0000: push_byte          01
% 0002: push_byte          02
% 0004: push_function      0000
% 0007: invoke            
% 0008: pop               
% \end{lstlisting}
% Functions:
% \vspace{-.1cm}
% \begin{lstlisting}[numbers=none]
% 0000: max (integer a, integer b) -> integer
% \end{lstlisting}
%         \end{minipage}
%     }
%     ;

%     \draw [arrow1] (lp) 
%     -- node [above] {代码} node [below] {生成}
%     (rp);
% \end{tikzpicture} 

%%%%%%%%%%%%%%%%%%%%%%%%%%%%%%% generator4.pdf
% \newcommand{\token}[1]{\texttt{\bfseries #1}}
% \begin{tikzpicture}[level distance=1.5cm,
%     level 1/.style={sibling distance=2.5cm, level distance=1.3cm},
%     level 2/.style={sibling distance=1.3cm, level distance=1cm}]
    
%     \coordinate (lp);
%     \coordinate [right = of lp] (rp);
%     \coordinate (lrt) at ($ (lp) + (-2.5cm, 1.7cm) $);
%     \coordinate (rrt) at ($ (rp) + (+2.7cm, 0cm) $);
%     \node at (lrt) {statements}
%     child{
%       node{class} 
%       child{node{\token{Obj}}} 
%       child{node{constructor} }
%     }
%     child{node{expression} child{node{\token{new}} child{node{\token{A}}}}}
%     ;
%     \node at (rrt) {
%         \begin{minipage}{4.5cm}
% Bytecode:
% \vspace{-.1cm}
% \begin{lstlisting}[numbers=none]
% 0000: new                0000
% 0003: duplicate_offset   0000
% 0006: push_method        0000
% 0009: invoke            
% 000A: pop               
% 000B: pop                        
% \end{lstlisting}
% Functions:
% \vspace{-.1cm}
% \begin{lstlisting}[numbers=none]
% 0000: A#initialize () -> void
% \end{lstlisting}
%         \end{minipage}
%     }
%     ;

%     \draw [arrow1] (lp) 
%     -- node [above] {代码} node [below] {生成}
%     (rp);
% \end{tikzpicture} 

%%%%%%%%%%%%%%%%%%%%%%%%%%%%%%% generator5.pdf
% \newcommand{\token}[1]{\texttt{\bfseries #1}}
% \begin{tikzpicture}[level distance=1.5cm,
%     level 1/.style={sibling distance=2cm, level distance=1.3cm},
%     level 2/.style={sibling distance=1.3cm, level distance=1cm}]
    
%     \coordinate (lp);
%     \coordinate [right = of lp] (rp);
%     \coordinate (lrt) at ($ (lp) + (-2cm, 1.5cm) $);
%     \coordinate (rrt) at ($ (rp) + (+2.7cm, 0cm) $);
%     \node at (lrt) {if\_statement}
%     child{node{$=$} child{node{1}} child{node{\token{a}}}}
%     child{node{block} child{node {$:$} child{node{\token{res}}} child{node{\texttt{"ok"}}}}}
%     ;
%     \node at (rrt) {
%         \begin{minipage}{4cm}
% Bytecode:
% \vspace{-.1cm}
% \begin{lstlisting}[numbers=none]
% 0000: push_byte          01
% 0002: push_static_int    0000
% 0005: eq_int            
% 0006: jump_if_false      0012
% 0009: push_str           0000
% 000C: pop_static_object  0001
% 000F: jump               0012
% \end{lstlisting}
% Constant Pool:
% \vspace{-.1cm}
% \begin{lstlisting}[numbers=none]
% 0000: "ok"
% \end{lstlisting}
% Static variables:
% \vspace{-.1cm}
% \begin{lstlisting}[numbers=none]
% 0000: integer a
% 0001: string res
% \end{lstlisting}
%         \end{minipage}
%     }
%     ;

%     \draw [arrow1] (lp) 
%     -- node [above] {代码} node [below] {生成}
%     (rp);
% \end{tikzpicture} 

%%%%%%%%%%%%%%%%%%%%%%%%%%%%%%% generator6.pdf
% \newcommand{\token}[1]{\texttt{\bfseries #1}}
% \begin{tikzpicture}[level distance=1.5cm,
%     level 1/.style={sibling distance=2cm, level distance=1.3cm},
%     level 2/.style={sibling distance=1.3cm, level distance=1cm}]
    
%     \coordinate (lp);
%     \coordinate [right = of lp] (rp);
%     \coordinate (lrt) at ($ (lp) + (-2cm, 0.65cm) $);
%     \coordinate (rrt) at ($ (rp) + (+2.7cm, 0cm) $);
%     \node at (lrt) {while\_statement}
%     child{node{\token{cond}}}
%     child{node{block}}
%     ;
%     \node at (rrt) {
%         \begin{minipage}{4cm}
% Bytecode:
% \vspace{-.1cm}
% \begin{lstlisting}[numbers=none]
% 0000: push_static_int    0000
% 0003: jump_if_false      0009
% 0006: jump               0000
% \end{lstlisting}
% Static variables:
% \vspace{-.1cm}
% \begin{lstlisting}[numbers=none]
% 0000: boolean cond
% \end{lstlisting}
%         \end{minipage}
%     }
%     ;

%     \draw [arrow1] (lp) 
%     -- node [above] {代码} node [below] {生成}
%     (rp);
% \end{tikzpicture} 

%%%%%%%%%%%%%%%%%%%%%%%%%%%%%%% generator7.pdf
% \newcommand{\token}[1]{\texttt{\bfseries #1}}
% \begin{tikzpicture}[level distance=1.5cm,
%     level 1/.style={sibling distance=2cm, level distance=1.3cm},
%     level 2/.style={sibling distance=1.3cm, level distance=1cm}]
    
%     \coordinate (lp);
%     \coordinate [right = of lp] (rp);
%     \coordinate (lrt) at ($ (lp) + (-2cm, 0.65cm) $);
%     \coordinate (rrt) at ($ (rp) + (+2.7cm, 0cm) $);
%     \node at (lrt) {repeat\_statement}
%     child{node{2}}
%     child{node{block}}
%     ;
%     \node at (rrt) {
%         \begin{minipage}{4cm}
% Bytecode:
% \vspace{-.1cm}
% \begin{lstlisting}[numbers=none]
% 0000: push_byte          02
% 0002: duplicate         
% 0003: jump_if_false      000A
% 0006: decrement         
% 0007: jump               0002
% 000A: pop               
% \end{lstlisting}
%         \end{minipage}
%     }
%     ;
%     \draw [arrow1] (lp) 
%     -- node [above] {代码} node [below] {生成}
%     (rp);
% \end{tikzpicture} 

%%%%%%%%%%%%%%%%%%%%%%%%%%%%%%% generator8.pdf
% \newcommand{\token}[1]{\texttt{\bfseries #1}}
% \begin{tikzpicture}[level distance=1.5cm,
%     level 1/.style={sibling distance=2cm, level distance=1.3cm},
%     level 2/.style={sibling distance=1.3cm, level distance=1cm}]
    
%     \coordinate (lp);
%     \coordinate [right = of lp] (rp);
%     \coordinate (lrt) at ($ (lp) + (-3.3cm, 1.3cm) $);
%     \coordinate (rrt) at ($ (rp) + (+2.7cm, 0cm) $);
%     \node at (lrt) {switch\_statement}
%     child{node{\token{t}}}
%     child{node{\token{case}} child{node{1}} child{node{block}}}
%     child{node{default\_block}}
%     ;
%     \node at (rrt) {
%         \begin{minipage}{4cm}
% Bytecode:
% \vspace{-.1cm}
% \begin{lstlisting}[numbers=none]
% 0000: push_static_int    0000
% 0003: duplicate         
% 0004: push_byte          01
% 0006: eq_int            
% 0007: jump_if_true       000D
% 000A: jump               0010
% 000D: jump               0010
% 0010: pop               
% \end{lstlisting}
% Static variables:
% \vspace{-.1cm}
% \begin{lstlisting}[numbers=none]
% 0000: integer t
% \end{lstlisting}
%         \end{minipage}
%     }
%     ;
%     \draw [arrow1] (lp) 
%     -- node [above] {代码} node [below] {生成}
%     (rp);
% \end{tikzpicture} 


%%%%%%%%%%%%%%%%%%%%%%%%%%%%%%% generator9.pdf
% \newcommand{\token}[1]{\texttt{\bfseries #1}}
% \begin{tikzpicture}[level distance=1.5cm,
%     level 1/.style={sibling distance=1.7cm, level distance=1.3cm},
%     level 2/.style={sibling distance=1.2cm, level distance=1cm}]
    
%     \coordinate (lp);
%     \coordinate [right = of lp] (rp);
%     \coordinate (lrt) at ($ (lp) + (-2.5cm, 1.3cm) $);
%     \coordinate (rrt) at ($ (rp) + (+2.7cm, 0cm) $);
%     \node at (lrt) {declare}
%     child{node{\token{integer}}}
%     child{node{\token{a}}}
%     child{node{$+$} child{node{\token{t}}} child{node{1}}}
%     ;
%     \node at (rrt) {
%         \begin{minipage}{4cm}
% Bytecode:
% \vspace{-.1cm}
% \begin{lstlisting}[numbers=none]
% 0000: push_static_int    0000
% 0003: push_byte          01
% 0005: add_int           
% 0006: pop_static_int     0001
% \end{lstlisting}
% Static variables:
% \vspace{-.1cm}
% \begin{lstlisting}[numbers=none]
% 0000: integer t
% 0001: integer a
% \end{lstlisting}
%         \end{minipage}
%     }
%     ;

%     \draw [arrow1] (lp) 
%     -- node [above] {代码} node [below] {生成}
%     (rp);
% \end{tikzpicture} 


%%%%%%%%%%%%%%%%%%%%%%%%%%%%%%% e_optimizer.pdf
% \newcommand{\token}[1]{\texttt{\bfseries #1}}
% \begin{tikzpicture}[level distance=1.5cm,
%     level 1/.style={sibling distance=2cm, level distance=1.3cm},
%     level 2/.style={sibling distance=1.3cm, level distance=1cm}]
%         \node [rect1] (optimizer) {代码优化子程序};
%         \coordinate [left = 2cm of optimizer] (lp);
%         \coordinate [right = 2cm of optimizer] (rp);
%         \coordinate (lrt) at ($ (lp) + (-2.3cm, 0cm) $);
%         \coordinate (rrt) at ($ (rp) + (+2.3cm, 0cm) $);
    
%     \node at (lrt) {
%         \begin{minipage}{4cm}
% Bytecode:
% \vspace{-.1cm}
% \begin{lstlisting}[numbers=none]
% 0000: push_float         0000
% 0003: push_byte          02
% 0005: cast_int_to_float 
% 0006: add_float         
% 0007: pop               
% \end{lstlisting}
% Constant Pool:
% \vspace{-.1cm}
% \begin{lstlisting}[numbers=none]
% 0000: 1.5
% \end{lstlisting}
%         \end{minipage}
%     }
%     ;
%     \node at (rrt) {
%         \vspace{.3cm}
%         \begin{minipage}{4cm}
% Bytecode
% \begin{lstlisting}[numbers=none]
% 0000: push_float         0000   
% 0003: pop               
% \end{lstlisting}
% Constant Pool:
% \vspace{-.1cm}
% \begin{lstlisting}[numbers=none]
% 0000: 3.5
% \end{lstlisting}
%         \end{minipage}
%     }
%     ;
%     % 绘制中间连线
%     \draw [arrow1] (lp) 
%     -- node [above] {字节码}
%     (optimizer);
%     \draw [arrow1] (optimizer)
%     -- node [above] {优化后的} node [below] {字节码}
%     (rp);
% \end{tikzpicture} 


%%%%%%%%%%%%%%%%%%%%%%%%%%%%%%% optimizer1.pdf
% \newcommand{\token}[1]{\texttt{\bfseries #1}}
% \begin{tikzpicture}[level distance=1.5cm,
%     level 1/.style={sibling distance=2cm, level distance=1.3cm},
%     level 2/.style={sibling distance=1.3cm, level distance=1cm}]
    
%     \coordinate (lp);
%     \coordinate [right = 2.3cm of lp] (rp);
%     \coordinate (lrt) at ($ (lp) + (-2cm, 1.3cm) $);
%     \coordinate (rrt) at ($ (rp) + (+2cm, 1.3cm) $);
%     \node at (lrt) {$-$}
%     child{node{$+$} child{node{1}} child{node{$*$} child{node{2}} child{node{3}}}}
%     child{node{t}}
%     ;
%     \node at (rrt) {$-$}
%     child{node{7}}
%     child{node{t}}
%     ;

%     \draw [arrow1] (lp) -- node [above] {常量折叠} (rp);
% \end{tikzpicture} 


%%%%%%%%%%%%%%%%%%%%%%%%%%%%%%% optimizer2.pdf
% \newcommand{\token}[1]{\texttt{\bfseries #1}}
% \begin{tikzpicture}[level distance=1.5cm,
%     level 1/.style={sibling distance=2cm, level distance=1.3cm},
%     level 2/.style={sibling distance=1.3cm, level distance=1cm}]
    
%     \coordinate (lp);
%     \coordinate [right = 2.3cm of lp] (rp);
%     \coordinate (lrt) at ($ (lp) + (-3.5cm, 1.3cm) $);
%     \coordinate (rrt) at ($ (rp) + (+2.5cm, 1.3cm) $);
%     \node at (lrt) {\token{block}}
%     child{node{\token{println}} child{node{\texttt{"before return"}}}}
%     child{node{\token{return}} child{node{0}}}
%     child{node{\token{println}} child{node{\texttt{"after return"}}}}
%     ;
%     \node at (rrt) {\token{block}}
%     child{node{\token{println}} child{node{\texttt{"before return"}}}}
%     child{node{\token{return}} child{node{0}}}
%     ;

%     \draw [arrow1] (lp) -- node [above] {死码消除} (rp);
% \end{tikzpicture} 


%%%%%%%%%%%%%%%%%%%%%%%%%%%%%%% optimizer3.pdf
% \newcommand{\token}[1]{\texttt{\bfseries #1}}
% \begin{tikzpicture}[level distance=1.5cm,
%     level 1/.style={sibling distance=2cm, level distance=1.3cm},
%     level 2/.style={sibling distance=2cm, level distance=1cm},
%     level 3/.style={sibling distance=1.3cm, level distance=1cm}]
    
%     \coordinate (lp);
%     \coordinate [right = 2.3cm of lp] (rp);
%     \coordinate (lrt) at ($ (lp) + (-2.5cm, 1.3cm) $);
%     \coordinate (rrt) at ($ (rp) + (+2cm, 1.3cm) $);
%     \node at (lrt) {\token{outer\_block}}
%     child {
%         node{\token{repeat}}
%         child{node{$-$} child{node{3}} child{node{2}}}
%         child{node{\token{inner\_block}} child{node{略}}}
%     }
%     ;
%     \node at (rrt) {\token{outer\_block}}
%     child {node{\token{inner\_block}} child{node{略}}}
%     ;

%     \draw [arrow1] (lp) -- node [above] {常量折叠} node [below] {repeat 展开} (rp);
% \end{tikzpicture} 


%%%%%%%%%%%%%%%%%%%%%%%%%%%%%%% optimizer5.pdf
% \newcommand{\token}[1]{\texttt{\bfseries #1}}
% \begin{tikzpicture}[level distance=1.5cm,
%     level 1/.style={sibling distance=2cm, level distance=1.3cm},
%     level 2/.style={sibling distance=1.3cm, level distance=1cm}]

%         \coordinate (lp);
%         \coordinate [right = 1.5cm of lp] (rp);
%         \coordinate (lrt) at ($ (lp) + (-2.3cm, 0cm) $);
%         \coordinate (rrt) at ($ (rp) + (+2.3cm, 0cm) $);
    
%     \node at (lrt) {
%         \begin{minipage}{4cm}
% Bytecode:
% \vspace{-.1cm}
% \begin{lstlisting}[numbers=none]
% 0000: push_byte    01
% 0002: push_byte    02
% 0004: push_byte    03
% 0006: mul_int           
% 0007: add_int           
% 0008: push_static_int    0000
% 000B: sub_int           
% 000C: pop               
% \end{lstlisting}
% Static variables:
% \vspace{-.1cm}
% \begin{lstlisting}[numbers=none]
% 0000: integer t
% \end{lstlisting}
%         \end{minipage}
%     }
%     ;
%     \node at (rrt) {
%         \vspace{.3cm}
%         \begin{minipage}{4cm}
% Bytecode:
% \vspace{-.1cm}
% \begin{lstlisting}[numbers=none]
% 0000: push_byte    07
% 0002: push_static_int    0000
% 0005: sub_int           
% 0006: pop               
% \end{lstlisting}
% Static variables:
% \vspace{-.1cm}
% \begin{lstlisting}[numbers=none]
% 0000: integer t
% \end{lstlisting}
%         \end{minipage}
%     }
%     ;
%     % 绘制中间连线
%     \draw [arrow1] (lp) 
%     -- node [above] {优化后}
%     (rp);
% \end{tikzpicture} 


%%%%%%%%%%%%%%%%%%%%%%%%%%%%%%% optimizer6.pdf
% \newcommand{\token}[1]{\texttt{\bfseries #1}}
% \begin{tikzpicture}[level distance=1.5cm,
%     level 1/.style={sibling distance=2cm, level distance=1.3cm},
%     level 2/.style={sibling distance=1.3cm, level distance=1cm}]

%         \coordinate (lp);
%         \coordinate [right = 1.5cm of lp] (rp);
%         \coordinate (lrt) at ($ (lp) + (-2.3cm, 0cm) $);
%         \coordinate (rrt) at ($ (rp) + (+2.3cm, 0cm) $);
    
%     \node at (lrt) {
%         \begin{minipage}{4cm}
% Bytecode:
% \vspace{-.1cm}
% \begin{lstlisting}[numbers=none]
% 0000: push_static_int    0000
% 0003: push_byte    02
% 0005: push_byte    03
% 0007: mul_int           
% 0008: add_int           
% 0009: push_byte    04
% 000B: sub_int           
% 000C: pop               
% \end{lstlisting}
% Static variables:
% \vspace{-.1cm}
% \begin{lstlisting}[numbers=none]
% 0000: integer t
% \end{lstlisting}
%         \end{minipage}
%     }
%     ;
%     \node at (rrt) {
%         \vspace{.3cm}
%         \begin{minipage}{4cm}
% Bytecode:
% \vspace{-.1cm}
% \begin{lstlisting}[numbers=none]
% 0000: push_static_int    0000
% 0003: push_byte    06    
% 0005: add_int           
% 0006: push_byte    04
% 0008: sub_int           
% 0009: pop               
% \end{lstlisting}
% Static variables:
% \vspace{-.1cm}
% \begin{lstlisting}[numbers=none]
% 0000: integer t
% \end{lstlisting}
%         \end{minipage}
%     }
%     ;
%     % 绘制中间连线
%     \draw [arrow1] (lp) 
%     -- node [above] {优化后}
%     (rp);
% \end{tikzpicture} 


%%%%%%%%%%%%%%%%%%%%%%%%%%%%%%% optimizer7.pdf
% \newcommand{\token}[1]{\texttt{\bfseries #1}}
% \begin{tikzpicture}[level distance=1.5cm,
%     level 1/.style={sibling distance=2cm, level distance=1.3cm},
%     level 2/.style={sibling distance=1.3cm, level distance=1cm}]

%         \coordinate (lp);
%         \coordinate [right = 1.5cm of lp] (rp);
%         \coordinate (lrt) at ($ (lp) + (-2.3cm, 0cm) $);
%         \coordinate (rrt) at ($ (rp) + (+2.3cm, 0cm) $);
    
%     \node at (lrt) {
%         \begin{minipage}{4cm}
% Bytecode:
% \vspace{-.1cm}
% \begin{lstlisting}[numbers=none]
% 0000: push_byte          0000
% 0003: jump_if_false      000A
% 0006: push_str           0000
% 0009: pop               
% 000A: push_static_int    0000
% 000D: jump_if_false      0014
% 0010: push_str           0001
% 0013: pop               
% 0014: push_byte          0001
% 0017: jump_if_false      001E
% 001A: push_str           0002
% 001D: pop               
% \end{lstlisting}
% Constant Pool:
% \vspace{-.1cm}
% \begin{lstlisting}[numbers=none]
% 0000: "false_block"
% 0001: "conditional_block"
% 0002: "true_block"
% \end{lstlisting}
% Static variables:
% \vspace{-.1cm}
% \begin{lstlisting}[numbers=none]
% 0000: boolean cond
% \end{lstlisting}
%         \end{minipage}
%     }
%     ;
%     \node at (rrt) {
%         \vspace{.3cm}
%         \begin{minipage}{4cm}
% Bytecode:
% \vspace{-.1cm}
% \begin{lstlisting}[numbers=none]
% 0000: push_static_int    0000
% 0003: jump_if_false      000A
% 0006: push_str           0000
% 0009: pop               
% 000A: push_str           0001
% 000D: pop               
% \end{lstlisting}
% Constant Pool:
% \vspace{-.1cm}
% \begin{lstlisting}[numbers=none]
% 0000: "conditional_block"
% 0001: "true_block"
% \end{lstlisting}
% Static variables:
% \vspace{-.1cm}
% \begin{lstlisting}[numbers=none]
% 0000: boolean cond
% \end{lstlisting}
%         \end{minipage}
%     }
%     ;
%     % 绘制中间连线
%     \draw [arrow1] (lp) 
%     -- node [above] {优化后}
%     (rp);
% \end{tikzpicture} 


%%%%%%%%%%%%%%%%%%%%%%%%%%%%%%% optimizer8.pdf
% \newcommand{\token}[1]{\texttt{\bfseries #1}}
% \begin{tikzpicture}[level distance=1.5cm,
%     level 1/.style={sibling distance=2cm, level distance=1.3cm},
%     level 2/.style={sibling distance=1.3cm, level distance=1cm}]

%         \coordinate (lp);
%         \coordinate [right = 1.5cm of lp] (rp);
%         \coordinate (lrt) at ($ (lp) + (-2.3cm, 0cm) $);
%         \coordinate (rrt) at ($ (rp) + (+2.3cm, 0cm) $);
    
%     \node at (lrt) {
%         \begin{minipage}{4cm}
% Bytecode:
% \vspace{-.1cm}
% \begin{lstlisting}[numbers=none]
% 0000: push_byte          01
% 0002: jump_if_false      0013
% 0005: push_str           0000
% 0008: pop               
% 0009: jump               0013
% 000C: push_str           0001
% 000F: pop               
% 0010: jump               0000
% \end{lstlisting}
% Constant Pool:
% \vspace{-.1cm}
% \begin{lstlisting}[numbers=none]
% 0000: "before_break"
% 0001: "after_break"
% \end{lstlisting}
%         \end{minipage}
%     }
%     ;
%     \node at (rrt) {
%         \vspace{.3cm}
%         \begin{minipage}{4cm}
% Bytecode:
% \vspace{-.1cm}
% \begin{lstlisting}[numbers=none]
% 0000: push_byte          01
% 0002: jump_if_false      000F
% 0005: push_str           0000
% 0008: pop               
% 0009: jump               000F
% 000C: jump               0000
% \end{lstlisting}
% Constant Pool:
% \vspace{-.1cm}
% \begin{lstlisting}[numbers=none]
% 0000: "before_break"
% \end{lstlisting}
%         \end{minipage}
%     }
%     ;
%     % 绘制中间连线
%     \draw [arrow1] (lp) 
%     -- node [above] {优化后}
%     (rp);
% \end{tikzpicture} 


%%%%%%%%%%%%%%%%%%%%%%%%%%%%%%% optimizer9.pdf
% \newcommand{\token}[1]{\texttt{\bfseries #1}}
% \begin{tikzpicture}[level distance=1.5cm,
%     level 1/.style={sibling distance=2cm, level distance=1.3cm},
%     level 2/.style={sibling distance=1.3cm, level distance=1cm}]

%         \coordinate (lp);
%         \coordinate [right = 1.5cm of lp] (rp);
%         \coordinate (lrt) at ($ (lp) + (-2.3cm, 0cm) $);
%         \coordinate (rrt) at ($ (rp) + (+2.3cm, 0cm) $);
    
%     \node at (lrt) {
%         \begin{minipage}{4cm}
% Bytecode:
% \vspace{-.1cm}
% \begin{lstlisting}[numbers=none]
% 0000: push_byte          01
% 0002: jump_if_false      0013
% 0005: push_str           0000
% 0008: pop               
% 0009: jump               0013
% 000C: push_str           0001
% 000F: pop               
% 0010: jump               0000
% \end{lstlisting}
% Constant Pool:
% \vspace{-.1cm}
% \begin{lstlisting}[numbers=none]
% 0000: "before_break"
% 0001: "after_break"
% \end{lstlisting}
%         \end{minipage}
%     }
%     ;
%     \node at (rrt) {
%         \vspace{.3cm}
%         \begin{minipage}{4cm}
% Bytecode:
% \vspace{-.1cm}
% \begin{lstlisting}[numbers=none]
% 0000: push_byte          01
% 0002: jump_if_false      000F
% 0005: push_str           0000
% 0008: pop               
% 0009: jump               000F
% 000C: jump               0000
% \end{lstlisting}
% Constant Pool:
% \vspace{-.1cm}
% \begin{lstlisting}[numbers=none]
% 0000: "before_break"
% \end{lstlisting}
%         \end{minipage}
%     }
%     ;
%     % 绘制中间连线
%     \draw [arrow1] (lp) 
%     -- node [above] {优化后}
%     (rp);
% \end{tikzpicture} 


%%%%%%%%%%%%%%%%%%%%%%%%%%%%%%% optimizer10.pdf
% \newcommand{\token}[1]{\texttt{\bfseries #1}}
% \begin{tikzpicture}[level distance=1.5cm,
%     level 1/.style={sibling distance=2cm, level distance=1.3cm},
%     level 2/.style={sibling distance=1.3cm, level distance=1cm}]

%         \coordinate (lp);
%         \coordinate [right = 1.5cm of lp] (rp);
%         \coordinate (lrt) at ($ (lp) + (-2.3cm, 0cm) $);
%         \coordinate (rrt) at ($ (rp) + (+2.3cm, 0cm) $);
    
%     \node at (lrt) {
%         \begin{minipage}{4cm}
% Bytecode:
% \vspace{-.1cm}
% \begin{lstlisting}[numbers=none]
% 0000: push_str           0000
% 0003: pop               
% 0004: push_str           0001
% 0007: return            
% 0008: push_str           0002
% 000B: pop               
% \end{lstlisting}
% Constant Pool:
% \vspace{-.1cm}
% \begin{lstlisting}[numbers=none]
% 0000: "before return"
% 0001: "return value"
% 0002: "after return"
% \end{lstlisting}
%         \end{minipage}
%     }
%     ;
%     \node at (rrt) {
%         \vspace{.3cm}
%         \begin{minipage}{4cm}
% Bytecode:
% \vspace{-.1cm}
% \begin{lstlisting}[numbers=none]
% 0000: push_str           0000
% 0003: pop               
% 0004: push_str           0001
% 0007: return            
% \end{lstlisting}
% Constant Pool:
% \vspace{-.1cm}
% \begin{lstlisting}[numbers=none]
% 0000: "before return"
% 0001: "return value"
% \end{lstlisting}
%         \end{minipage}
%     }
%     ;
%     % 绘制中间连线
%     \draw [arrow1] (lp) 
%     -- node [above] {优化后}
%     (rp);
% \end{tikzpicture} 


%%%%%%%%%%%%%%%%%%%%%%%%%%%%%%% detail_scanner1.pdf
% \begin{minipage}{8cm}
% \vspace{-.1cm}
% \begin{lstlisting}[numbers=none]
%   /* 保留字识别 */
% <INITIAL>"if"           return make_IF(loc);
% <INITIAL>"else"         return make_ELSE(loc);
% <INITIAL>"switch"       return make_SWITCH(loc);
% <INITIAL>"case"         return make_CASE(loc);
%  (其它保留字的识别代码类似,此处略去)
%  /* 标识符识别 */
% <INITIAL>[A-Za-z_][A-Za-z_0-9]* {
%     return make_IDENTIFIER(std::string(YYText()), loc);
% }
% \end{lstlisting}
% \vspace{-.34cm}
% \end{minipage}


%%%%%%%%%%%%%%%%%%%%%%%%%%%%%%% detail_scanner2.pdf
% \begin{minipage}{8cm}
% \vspace{-.1cm}
% \begin{lstlisting}[numbers=none]
% <INITIAL>0[xX](_?[0-9a-fA-F])+ {
%     IntExpression  *expression = new IntExpression;
%     int *x = &expression->value;
%     *x = 0;
%     for (int i = 2; yytext[i]; ++i) {
%         if (yytext[i] == '_') continue;
%         if (yytext[i] <= '9') 
%             *x = *x * 16 + (yytext[i] - '0');
%         else if (yytext[i] <= 'Z') 
%             *x = *x * 16 + (yytext[i] - 'A' + 10);
%         else *x = *x * 16 + (yytext[i] - 'a' + 10);
%     }
%     return make_INT_LITERAL(expression, loc);
% }
% \end{lstlisting}
% \vspace{-.34cm}
% \end{minipage}


%%%%%%%%%%%%%%%%%%%%%%%%%%%%%%% detail_scanner3.pdf
% \begin{minipage}{8cm}
% \vspace{-.1cm}
% \begin{lstlisting}[numbers=none]
% <INITIAL>([0-9]_?)*[0-9]\.(([0-9]_?)*[0-9])? {
%     FloatExpression  *expression = new FloatExpression;
%     int i = 0, j = -1;
%     while (yytext[++j]) {
%         if (yytext[j] == '_') continue;
%         yytext[i++] = yytext[j];
%     }
%     yytext[i] = '\0';
%     sscanf(yytext, "%lf", &expression->value);
%     return make_FLOAT_LITERAL(expression, loc);
% }
% \end{lstlisting}
% \vspace{-.34cm}
% \end{minipage}


%%%%%%%%%%%%%%%%%%%%%%%%%%%%%%% detail_scanner4.pdf
% \begin{minipage}{8cm}
% \vspace{-.1cm}
% \begin{lstlisting}[numbers=none]
% <INITIAL>\' BEGIN STRING_S;
% <STRING_S>{
%   \n      { throw syntax_error(loc, 
%               "unexpected newline in a string."); }
%   <<EOF>> { throw syntax_error(loc, 
%               "unexpected end of file."); }
%   '       {
%                StrExpression *expression = StrExpression;
%                expression->value = tmp_s;
%                BEGIN INITIAL;
%                return make_STRING_LITERAL(expression, loc);
%           }
%   \\\\    tmp_s += '\\';
%   \\t     tmp_s += '\t';
%   \\n     tmp_s += "\r\n";
%   \\'     tmp_s += '\'';
%   \\.     tmp_s += yytext[1];
%   .       tmp_s += yytext[0];
% }
% \end{lstlisting}
% \vspace{-.34cm}
% \end{minipage}


%%%%%%%%%%%%%%%%%%%%%%%%%%%%%%% detail_scanner5.pdf
% \begin{minipage}{8cm}
% \vspace{-.1cm}
% \begin{lstlisting}[numbers=none]
% <INITIAL>"(*"     depthOfComment = 1, BEGIN BO_COMMENT;
% <BO_COMMENT>{
%     \n     loc.lines(yyleng); loc.step ();
%     "(*"   ++depthOfComment;
%     "*)"   if (--depthOfComment == 0) BEGIN INITIAL;
%     <<EOF>> {
%       throw syntax_error(loc, "unexpected end of line.");
%     }
%     .      ;
% }
% \end{lstlisting}
% \vspace{-.34cm}
% \end{minipage}


%%%%%%%%%%%%%%%%%%%%%%%%%%%%%%% detail_scanner6.pdf
% \begin{minipage}{8cm}
% \vspace{-.1cm}
% \begin{lstlisting}[numbers=none]
% <INITIAL>"("            return make_LP(loc);
% <INITIAL>")"            return make_RP(loc);
% <INITIAL>"{"            return make_LC(loc);
% <INITIAL>"}"            return make_RC(loc);
% <INITIAL>"["            return make_LB(loc);
% <INITIAL>"]"            return make_RB(loc);
% <INITIAL>";"            return make_SEMICOLON(loc);
% <INITIAL>":"            return make_ASSIGN(loc);
% <INITIAL>"="            return make_EQ(loc);
% <INITIAL>"!="           return make_NE(loc);
%  (其它运算符的识别代码类似,此处略去)
% \end{lstlisting}
% \vspace{-.34cm}
% \end{minipage}


%%%%%%%%%%%%%%%%%%%%%%%%%%%%%%% detail_parser.pdf
% \begin{minipage}{9cm}
% \vspace{-.1cm}
% \begin{lstlisting}[numbers=none]
% %start bo_program;
% bo_program
%   : initial_declaration
%   | translation_unit
%   ;
% translation_unit
%   : initial_declaration definition_or_statement
%   | translation_unit definition_or_statement
%   ;
% \end{lstlisting}
% \vspace{-.34cm}
% \end{minipage}


%%%%%%%%%%%%%%%%%%%%%%%%%%%%%%% detail_parser1.pdf
% \begin{minipage}{9cm}
% \vspace{-.1cm}
% \begin{lstlisting}[numbers=none]
% %left ASSIGN_T;
% %left ADD SUB;
% %left MUL DIV MOD;
% %precedence UMINUS;
% %nonassoc INSTANCEOF;
% expression
%   : primary_expression
%   | LB boolean_expression RB { $$ = $2; }
%   | expression ASSIGN_T expression 
%   { $$ = create_assign_expression($1, NORMAL_ASSIGN, $3); }
%   | expression ADD expression 
%   { $$ = create_binary_expression(ADD_EXPRESSION, $1, $3); }
%   | expression SUB expression 
%   { $$ = create_binary_expression(SUB_EXPRESSION, $1, $3); }
%   | expression MUL expression 
%   { $$ = create_binary_expression(MUL_EXPRESSION, $1, $3); }
%   (其它运算符的识别规则和语义动作类似,此处略去)
%   ;
% \end{lstlisting}
% \vspace{-.34cm}
% \end{minipage}


%%%%%%%%%%%%%%%%%%%%%%%%%%%%%%% detail_parser2.pdf
% \begin{minipage}{9cm}
% \vspace{-.1cm}
% \begin{lstlisting}[numbers=none]
% %left LOGICAL_OR;
% %left LOGICAL_AND;
% %left EQ NE GT GE LT LE;
% boolean_expression
%   : expression 
%   { $1->is_relational_expression = false; $$ = $1; }
%   | boolean_expression LOGICAL_OR  boolean_expression { $$ = 
%     create_binary_expression(LOGICAL_OR_EXPRESSION, $1, $3); }
%   | boolean_expression LOGICAL_AND boolean_expression { $$ = 
%     create_binary_expression(LOGICAL_AND_EXPRESSION, $1, $3); }
%   | boolean_expression EQ boolean_expression { $$ = 
%     create_binary_expression(EQ_EXPRESSION, $1, $3); }
%   (其它运算符的识别规则和语义动作类似,此处略去)
%   ;
% \end{lstlisting}
% \vspace{-.34cm}
% \end{minipage}


%%%%%%%%%%%%%%%%%%%%%%%%%%%%%%% detail_parser3.pdf
% \begin{minipage}{9cm}
% \vspace{-.1cm}
% \begin{lstlisting}[numbers=none]
% if_statement
%   : IF expression block { $$ = 
%     create_if_statement($2, $3, nullptr, nullptr); }
%   | IF expression block ELSE block { $$ = 
%     create_if_statement($2, $3, nullptr, $5); }
%   | IF expression block elsif_list { $$ = 
%     create_if_statement($2, $3, $4, nullptr); }
%   | IF expression block elsif_list ELSE block { $$ = 
%     create_if_statement($2, $3, $4, $6); };
% elsif_list
%   : elsif
%   | elsif_list elsif { $$ = chain_elsif_list($1, $2); };
% elsif
%   : ELSE IF expression block { $$ = create_elsif($3, $4); };
% \end{lstlisting}
% \vspace{-.34cm}
% \end{minipage}


%%%%%%%%%%%%%%%%%%%%%%%%%%%%%%% detail_parser4.pdf
% \begin{minipage}{9cm}
% \vspace{-.1cm}
% \begin{lstlisting}[numbers=none]
% switch_statement
%   : SWITCH LP expression RP case_list default_opt { $$ = 
%     create_switch_statement($3, $5, $6); };
% case_list : one_case
%   | case_list one_case { $$ = chain_case($1, $2); };
% one_case
%   : CASE case_expression_list block { $$ = 
%     create_one_case($2, $3); };
% default_opt : %empty { $$ = nullptr; }
%   | DEFAULT_T block { $$ = $2; };
% \end{lstlisting}
% \vspace{-.34cm}
% \end{minipage}


%%%%%%%%%%%%%%%%%%%%%%%%%%%%%%% detail_parser5.pdf
% \begin{minipage}{9cm}
% \vspace{-.1cm}
% \begin{lstlisting}[numbers=none]
% label_opt : %empty { $$ = ""; } 
%   | IDENTIFIER LINK { $$ = $1; };
% while_statement : label_opt WHILE expression block 
%   { $$ = create_while_statement($1, $3, $4); };
% repeat_statement : label_opt REPEAT expression block 
%   { $$ = create_repeat_statement($1, $3, $4); };
% \end{lstlisting}
% \vspace{-.34cm}
% \end{minipage}


%%%%%%%%%%%%%%%%%%%%%%%%%%%%%%% detail_parser6.pdf
% \begin{minipage}{9cm}
% \vspace{-.1cm}
% \begin{lstlisting}[numbers=none]
% declaration_statement
%   : type_specifier IDENTIFIER SEMICOLON 
%   { $$ = create_declaration_statement(false, $1, $2, nullptr); }
%   | type_specifier IDENTIFIER ASSIGN_T expression SEMICOLON 
%   { $$ = create_declaration_statement(false, $1, $2, $4); };
% \end{lstlisting}
% \vspace{-.34cm}
% \end{minipage}


%%%%%%%%%%%%%%%%%%%%%%%%%%%%%%% detail_parser7.pdf
% \begin{minipage}{9cm}
% \vspace{-.1cm}
% \begin{lstlisting}[numbers=none]
% function_definition
%   : DEFINE IDENTIFIER LP parameter_list RP
%    RETURN_TYPE_POINTER type_specifier throws_clause block
%   { function_define($7, $2, $4, $8, $9); }
%   | DEFINE IDENTIFIER LP RP 
%    RETURN_TYPE_POINTER type_specifier throws_clause block
%   { function_define($6, $2, nullptr, $7, $8); };
% \end{lstlisting}
% \vspace{-.34cm}
% \end{minipage}


%%%%%%%%%%%%%%%%%%%%%%%%%%%%%%% detail_parser8.pdf
% \begin{minipage}{9cm}
% \vspace{-.1cm}
% \begin{lstlisting}[numbers=none]
% class_definition
%   : class_or_interface IDENTIFIER extends 
%     LC { start_class_definition(nullptr, $1, $2, $3); }
%     member_declaration_list RC { class_define($6); }
%   | class_or_member_modifier_list class_or_interface IDENTIFIER
%     extends LC { start_class_definition(&$1, $2, $3, $4); } 
%     member_declaration_list RC { class_define($7); };
% class_or_member_modifier_list : class_or_member_modifier
%   | class_or_member_modifier_list class_or_member_modifier { $$ 
%     = chain_class_or_member_modifier($1, $2); };
% \end{lstlisting}
% \vspace{-.34cm}
% \end{minipage}


%%%%%%%%%%%%%%%%%%%%%%%%%%%%%%% detail_semantic11.pdf
% \begin{tikzpicture}
%     % Place nodes
%     \node [start] (init) {开始};
%     \node [process, below = of init] (s1) {对左侧表达式进行语义分析};
%     \node [decision, below = of s1] (d1) {类型是否正确?};
%     \node [process, below = of d1] (s2) {对右侧表达式进行语义分析};
%     \node [decision, below = of s2] (d2) {类型是否一致?};
%     \node [decision, below = of d2] (d3) {能否类型转换?};
%     \node [process, below = of d3] (s3) {添加类型转换结点};
%     \node [start, below = of s3] (stop) {结束};

%     \coordinate [left = of d3] (lp);
%     \node [process, right = of d3] (ce) {编译错误};

%     \path [arrow1] (init) -- (s1);
%     \path [arrow1] (s1) -- (d1);
%     \path [arrow1] (d1) -- node [right, near start] {Y} (s2);
%     \path [arrow1] (s2) -- (d2);
%     \path [arrow1] (d2) -- node [right, near start] {N} (d3);
%     \path [arrow1] (d3) -- node [right, near start] {Y} (s3);
%     \path [arrow1] (s3) -- (stop);

%     \path [arrow1] (d1) -| node [above, near start] {N} (ce);
%     \path [arrow1] (d3) -- node [above, near start] {N} (ce);
%     \path [arrow1] (ce) |- (stop);
%     \path [arrow1] (d2) -| node [above, near start] {Y} (lp) |- (stop);
% \end{tikzpicture}


%%%%%%%%%%%%%%%%%%%%%%%%%%%%%%% detail_semantic12.pdf
% \begin{tikzpicture}
%     % Place nodes
%     \node [start] (init) {开始};
%     \node [process, below = of init] (s1) {对左侧表达式进行语义分析};
%     \node [process, below = of s1] (s2) {对右侧表达式进行语义分析};
%     \node [decision, below = of s2] (d1) {类型是否一致?};
%     \node [decision, below = of d1] (d2) {能否类型转换?};
%     \node [process, below = of d2] (s3) {添加类型转换结点};
%     \node [start, below = of s3] (stop) {结束};

%     \coordinate [left = of d2] (lp);
%     \node [process, right = of d2] (ce) {编译错误};

%     \path [arrow1] (init) -- (s1);
%     \path [arrow1] (s1) -- (s2);
%     \path [arrow1] (s2) -- (d1);
%     \path [arrow1] (d1) -- node [right, near start] {N} (d2);
%     \path [arrow1] (d2) -- node [right, near start] {Y} (s3);
%     \path [arrow1] (s3) -- (stop);

%     \path [arrow1] (d1) -| node [above, near start] {Y} (lp) |- (stop);
%     \path [arrow1] (d2) -- node [above, near start] {N} (ce);
%     \path [arrow1] (ce) |- (stop);
    
% \end{tikzpicture}


%%%%%%%%%%%%%%%%%%%%%%%%%%%%%%% detail_semantic13.pdf
% \begin{tikzpicture}
%     % Place nodes
%     \node [start] (init) {开始};
%     \node [process, below = of init] (s0) {调整语法树结构};
%     \node [process, below = of s0] (s1) {左子树语义分析};
%     \node [process, below = of s1] (s2) {右子树语义分析};
%     \node [decision, below = of s2] (d1) {类型是否一致?};
%     \node [decision, below right = of d1] (d2) {能否类型转换?};
%     \node [process, below = of d2] (s3) {添加类型转换结点};
%     \node [start, below = 4.4cm of d1] (stop) {结束};

%     \node [process, right = of d2] (ce) {编译错误};

%     \path [arrow1] (init) -- (s0);
%     \path [arrow1] (s0) -- (s1);
%     \path [arrow1] (s1) -- (s2);
%     \path [arrow1] (s2) -- (d1);
%     \path [arrow1] (d1) -| node [above, near start] {N} (d2);
%     \path [arrow1] (d2) -- node [right, near start] {Y} (s3);
%     \path [arrow1] (s3) |- (stop);

%     \path [arrow1] (d1) -- node [right, near start] {Y} (stop);
%     \path [arrow1] (d2) -- node [above, near start] {N} (ce);
%     \path [arrow1] (ce) |- (stop);
    
%     % \node at ($(decide) + (4cm, -1.4cm)$) [process] (load) {依次编译所需外部文件};
% \end{tikzpicture}


%%%%%%%%%%%%%%%%%%%%%%%%%%%%%%% detail_semantic14.pdf
% \begin{tikzpicture}
%     % Place nodes
%     \node [start] (init) {开始};
%     \node [decision, below = of init] (d1) {函数已定义?};
%     \node [decision, below = of d1] (d2) {形参数量匹配?};
%     \node [process, below = of d2] (s1) {指向第一个参数};
%     \node [process, below = of s1] (s2) {当前参数语义分析};
%     \node [decision, below = of s2] (d3) {类型是否一致?};
%     \node [decision, below right = of d3] (d4) {能否类型转换?};
%     \node [process, below = of d4] (s3) {添加类型转换结点};
%     \node [decision, below = 4.4cm of d3] (d5) {是否还有参数?};
%     \node [process, left = of d3] (s4) {指向下一参数};
%     \node [start, below = of d5] (stop) {结束};

%     \coordinate (p1) at ($(s2) + (0cm, 1cm)$);
%     \node [process, right = of s3] (ce) {编译错误};

%     \path [arrow1] (init) -- (d1);
%     \path [arrow1] (d1) -- node [right, near start] {Y} (d2);
%     \path [arrow1] (d2) -- node [right, near start] {Y} (s1);
%     \path [arrow1] (s1) -- (s2);
%     \path [arrow1] (s2) -- (d3);
%     \path [arrow1] (d3) -- node [right, near start] {Y} (d5);
%     \path [arrow1] (d5) -- node [right, near start] {N} (stop);

%     \path [arrow1] (d3) -| node [above, near start] {N} (d4);
%     \path [arrow1] (d4) -- node [right, near start] {Y} (s3);
%     \path [arrow1] (s3) -| (d5);
%     \path [arrow1] (d5) -| node [above, near start] {Y} (s4);
%     \path [arrow1] (s4) |- (p1) -- (s2);

    
%     \path [arrow1] (d1) -| node [above, near start] {N} (ce);
%     \path [arrow1] (d2) -| node [above, near start] {N} (ce);
%     \path [arrow1] (d4) -| node [above, near start] {N} (ce);
%     \path [arrow1] (ce) |- (stop);
    
% \end{tikzpicture}


%%%%%%%%%%%%%%%%%%%%%%%%%%%%%%% detail_semantic21.pdf
% \begin{tikzpicture}
%     % Place nodes
%     \node [start] (init) {开始};
%     \node [process, below = of init] (s1) {对表达式语义分析};
%     \node [decision, below = of s1] (d1) {是布尔类型?};
%     \node [process, below = of d1] (s2) { if 块语义分析};
%     \node [process, below = of s2] (s3) { else 块语义分析};
%     \node [start, below = of s3] (stop) {结束};

%     \node [process, right = of s3] (ce) {编译错误};

%     \path [arrow1] (init) -- (s1);
%     \path [arrow1] (s1) -- (d1);
%     \path [arrow1] (d1) -- node [right, near start] {Y} (s2);
%     \path [arrow1] (d1) -| node [above, near start] {N} (ce);
%     \path [arrow1] (s2) -- (s3);
%     \path [arrow1] (s3) -- (stop);
%     \path [arrow1] (ce) |- (stop);
    
% \end{tikzpicture}


%%%%%%%%%%%%%%%%%%%%%%%%%%%%%%% detail_semantic22.pdf
% \begin{tikzpicture}
%     % Place nodes
%     \node [start] (init) {开始};
%     \node [process, below = of init] (s1) {对表达式语义分析};
%     \node [decision, below = of s1] (d1) {还有 case 块?};
%     \node [process, below = of d1] (s2) {对当前 case 的表达式进行语义分析};
%     \node [process, below = of s2] (s3) {对当前 case 的代码块进行语义分析};
%     \node [process, right = of s2] (s4) {对 default 代码块进行语义分析};
%     \node [start, below = of s4] (stop) {结束};

%     % \node [process, right = of s3] (ce) {编译错误};

%     \path [arrow1] (init) -- (s1);
%     \path [arrow1] (s1) -- (d1);
%     \path [arrow1] (d1) -- node [right, near start] {Y} (s2);
%     \path [arrow1] (d1) -| node [above, near start] {N} (s4);
%     \path [arrow1] (s2) -- (s3);
%     \path [arrow1] (s3) |- ($ (s3) + (-3cm, -2cm) $) |- ($ (d1) + (0cm, 1.5cm) $) -- (d1);
%     \path [arrow1] (s4) -- (stop);

% \end{tikzpicture}


%%%%%%%%%%%%%%%%%%%%%%%%%%%%%%% detail_semantic23.pdf
% \begin{tikzpicture}
%     % Place nodes
%     \node [start] (init) {开始};
%     \node [process, below = of init] (s1) {对表达式语义分析};
%     \node [decision, below = of s1] (d1) {是布尔类型?};
%     \node [process, below = of d1] (s2) { 代码块语义分析};
%     \node [start, below = of s2] (stop) {结束};

%     \node [process, right = of s2] (ce) {编译错误};

%     \path [arrow1] (init) -- (s1);
%     \path [arrow1] (s1) -- (d1);
%     \path [arrow1] (d1) -- node [right, near start] {Y} (s2);
%     \path [arrow1] (d1) -| node [above, near start] {N} (ce);
%     \path [arrow1] (s2) -- (stop);
%     \path [arrow1] (ce) |- (stop);
    
% \end{tikzpicture}


%%%%%%%%%%%%%%%%%%%%%%%%%%%%%%% detail_semantic24.pdf
% \begin{tikzpicture}
%     % Place nodes
%     \node [start] (init) {开始};
%     \node [process, below = of init] (s1) {对表达式语义分析};
%     \node [decision, below = of s1] (d1) {是整数类型?};
%     \node [process, below = of d1] (s2) { 代码块语义分析};
%     \node [start, below = of s2] (stop) {结束};

%     \node [process, right = of s2] (ce) {编译错误};

%     \path [arrow1] (init) -- (s1);
%     \path [arrow1] (s1) -- (d1);
%     \path [arrow1] (d1) -- node [right, near start] {Y} (s2);
%     \path [arrow1] (d1) -| node [above, near start] {N} (ce);
%     \path [arrow1] (s2) -- (stop);
%     \path [arrow1] (ce) |- (stop);
    
% \end{tikzpicture}


%%%%%%%%%%%%%%%%%%%%%%%%%%%%%%% detail_semantic25.pdf
% \begin{tikzpicture}
%     % Place nodes
%     \node [start] (init) {开始};
%     \node [decision, below = of init] (s1) {标识符未声明?};
%     \node [process, below = of s1] (s2) {为当前块添加声明};
%     \node [decision, below = of s2] (s3) {有初值表达式?};
%     \node [process, below = of s3] (s4) {对表达式语义分析};
%     \node [start, below = of s4] (stop) {结束};

%     \node [process, right = of s4] (ce) {编译错误};

%     \path [arrow1] (init) -- (s1);
%     \path [arrow1] (s1) -- node [right, near start] {Y} (s2);
%     \path [arrow1] (s1) -| node [above, near start] {N} (ce);
%     \path [arrow1] (s2) -- (s3);
%     \path [arrow1] (s3) -- node [right, near start] {Y} (s4);
%     \path [arrow1] (s3) -- node [above, near start] {N} ($ (s3) + (-3cm, 0cm) $) |- (stop);
%     \path [arrow1] (s4) -- (stop);
%     \path [arrow1] (ce) |- (stop);
    
% \end{tikzpicture}


%%%%%%%%%%%%%%%%%%%%%%%%%%%%%%% detail_generator11.pdf
% \begin{tikzpicture}
%     % Place nodes
%     \node [start] (init) {开始};
%     \node [process, below = of init] (s1) {为左子树生成代码};
%     \node [process, below = of s1] (s2) {为右子树生成代码};
%     \node [process, below = of s2] (s3) {为操作符生成代码};
%     \node [start, below = of s3] (stop) {结束};

%     \path [arrow1] (init) -- (s1);
%     \path [arrow1] (s1) --  (s2);
%     \path [arrow1] (s2) -- (s3);
%     \path [arrow1] (s3) -- (stop);
    
% \end{tikzpicture}


%%%%%%%%%%%%%%%%%%%%%%%%%%%%%%% detail_generator12.pdf
% \begin{tikzpicture}
%     % Place nodes
%     \node [start] (init) {开始};
%     \node [process, below = of init] (s1) {为操作数生成代码};
%     \node [process, below = of s1] (s2) {为操作符生成代码};
%     \node [start, below = of s2] (stop) {结束};

%     \path [arrow1] (init) -- (s1);
%     \path [arrow1] (s1) --  (s2);
%     \path [arrow1] (s2) -- (stop);
    
% \end{tikzpicture}


%%%%%%%%%%%%%%%%%%%%%%%%%%%%%%% detail_generator13.pdf
% \begin{tikzpicture}
%     % Place nodes
%     \node [start] (init) {开始};
%     \node [process, below = of init] (s1) {i = 1};
%     \node [decision, below = of s1] (s2) {i <= 形参数量?};
%     \node [process, below = of s2] (s3) {为第 i 个参数生成代码};
%     \node [process] at ( $(s2) + (-3cm, -1cm)$ ) (s4) {i = i + 1};
%     \node [process, right = of s3] (s5) {生成函数调用代码};
%     \node [start, below = of s5] (stop) {结束};

%     \path [arrow1] (init) -- (s1);
%     \path [arrow1] (s1) --  (s2);
%     \path [arrow1] (s2) -- node [right, near start] {Y} (s3);
%     \path [arrow1] (s2) -| node [above, near start] {N} (s5);
%     \path [arrow1] (s3) -| (s4);
%     \path [arrow1] (s4) |- ($(s2)+(0cm,1.5cm)$) -- (s2);
%     \path [arrow1] (s5) -- (stop);
    
% \end{tikzpicture}


%%%%%%%%%%%%%%%%%%%%%%%%%%%%%%% detail_generator21.pdf
% \begin{tikzpicture}
%     % Place nodes
%     \node [start] (init) {开始};
%     \node [process, below = of init] (s1) {为表达式生成代码};
%     \node [process, below = of s1] (s2) {获取endif标签、生成条件跳转代码};
%     \node [process, below = of s2] (s3) {为 if 块生成代码};
%     \node [decision, below = of s3] (s4) {存在 else 块?};
%     \node [process, below left = of s4] (s5) {获取endelse标签、生成无条件跳转代码};
%     \node [process, below = of s5] (s6) {else 块生成代码};
%     \node [process, below right = of s4] (s7) {将标签代码替换为真正的地址};
%     \node [start, below = of s7] (stop) {结束};

%     \path [arrow1] (init) -- (s1);
%     \path [arrow1] (s1) --  (s2);
%     \path [arrow1] (s2) --  (s3);
%     \path [arrow1] (s3) --  (s4);
%     \path [arrow1] (s4) -| node [above, near start] {Y} (s5);
%     \path [arrow1] (s4) -| node [above, near start] {N} (s7);
%     \path [arrow1] (s5) --  (s6);
%     \path [arrow1] (s6) -- ($(s6)+(3.4cm,0cm)$) |- (s7);
%     \path [arrow1] (s7) -- (stop);
    
% \end{tikzpicture}


%%%%%%%%%%%%%%%%%%%%%%%%%%%%%%% detail_generator22.pdf
% \begin{tikzpicture}
%     % Place nodes
%     \node [start] (init) {开始};
%     \node [process, below = of init] (s1) {为表达式生成代码};
%     \node [process, below = of s1] (s2) {获取end标签};
%     \node [process, below = of s2] (s3) {i = 1};
%     \node [decision, below = of s3] (s4) {i <= case数目};
%     \node [process, below = of s4] (s5) {为case值生成代码};
%     \node [process, below = of s5] (s6) {获取endcase标签、生成条件跳转指令};
%     \node [process, below = of s6] (s7) {case块生成代码};
%     \node [process, below = of s7] (s8) {将endcase替换为真正的地址};
%     \node [process, below = of s8] (s9) {生成无条件跳转指令};
%     \node [process, left = of s8] (s10) {i = i + 1};
%     \node [decision, right = of s5] (s11) {存在default块?};
%     \node [process, below = of s11] (s12) {获取end标签、生成无条件跳转指令};
%     \node [process, below = of s12] (s13) {default块生成代码};
%     \node [process, below = of s13] (s14) {将end标签替换为真正的地址};
%     \node [start, below = of s14] (stop) {结束};

%     \path [arrow1] (init) -- (s1);
%     \path [arrow1] (s1) --  (s2);
%     \path [arrow1] (s2) --  (s3);
%     \path [arrow1] (s3) --  (s4);
%     \path [arrow1] (s4) -- node [right, near start] {Y} (s5);
%     \path [arrow1] (s5) --  (s6);
%     \path [arrow1] (s6) --  (s7);
%     \path [arrow1] (s7) --  (s8);
%     \path [arrow1] (s8) --  (s9);
%     \path [arrow1] (s9) -|  (s10);
%     \path [arrow1] (s10) |- ($(s4)+(0cm,1.5cm)$) -- (s4);
%     \path [arrow1] (s4) -| node [above, near start] {N} (s11);
%     \path [arrow1] (s11) -- node [right, near start] {Y} (s12);
%     \path [arrow1] (s12) -- (s13);
%     \path [arrow1] (s13) -- (s14);
%     \path [arrow1] (s14) -- (stop);
%     \path [arrow1] (s11) -| node [above, near start] {N} ($(stop)+(3cm, 0cm)$) -- (stop);
    
% \end{tikzpicture}


%%%%%%%%%%%%%%%%%%%%%%%%%%%%%%% detail_generator23.pdf
% \begin{tikzpicture}
%     % Place nodes
%     \node [start] (init) {开始};
%     \node [process, below = of init] (s1) {为表达式生成代码};
%     \node [process, below = of s1] (s2) {获取end标签、生成条件跳转指令};
%     \node [process, below = of s2] (s3) {为代码块生成代码};
%     \node [process, below = of s3] (s4) {生成无条件跳转指令};
%     \node [process, below = of s4] (s5) {将end标签替换为真正的地址};
%     \node [start, below = of s5] (stop) {结束};

%     \path [arrow1] (init) -- (s1);
%     \path [arrow1] (s1) --  (s2);
%     \path [arrow1] (s2) --  (s3);
%     \path [arrow1] (s3) --  (s4);
%     \path [arrow1] (s4) -- (s5);
%     \path [arrow1] (s5) -- (stop);
    
% \end{tikzpicture}


%%%%%%%%%%%%%%%%%%%%%%%%%%%%%%% detail_generator24.pdf
% \begin{tikzpicture}
%     % Place nodes
%     \node [start] (init) {开始};
%     \node [process, below = of init] (s1) {为表达式生成代码};
%     \node [process, below = of s1] (s2) {生成复制栈顶指令};
%     \node [process, below = of s2] (s3) {获取end标签、生成条件跳转指令};
%     \node [process, below = of s3] (s4) {为代码块生成代码};
%     \node [process, below = of s4] (s5) {生成栈顶减一指令};
%     \node [process, below = of s5] (s6) {生成无条件跳转指令};
%     \node [process, below = of s6] (s7) {生成弹出栈顶指令};
%     \node [process, below = of s7] (s8) {将end标签替换为真正的地址};
%     \node [start, below = of s8] (stop) {结束};

%     \path [arrow1] (init) -- (s1);
%     \path [arrow1] (s1) --  (s2);
%     \path [arrow1] (s2) --  (s3);
%     \path [arrow1] (s3) --  (s4);
%     \path [arrow1] (s4) -- (s5);
%     \path [arrow1] (s5) -- (s6);
%     \path [arrow1] (s6) -- (s7);
%     \path [arrow1] (s7) -- (s8);
%     \path [arrow1] (s8) -- (stop);
    
% \end{tikzpicture}


%%%%%%%%%%%%%%%%%%%%%%%%%%%%%%% detail_optimizer11.pdf
% \begin{tikzpicture}
%     % Place nodes
%     \node [start] (init) {开始};
%     \node [process, below = of init] (s1) {优化操作数1的代码};
%     \node [process, below = of s1] (s2) {优化操作数2的代码};
%     \node [decision, below = of s2] (s3) {操作数均为常量};
%     \node [process, below = of s3] (s4) {计算常量值、将操作数代码替换为当前常量};
%     \node [start, below = of s4] (stop) {结束};

%     \path [arrow1] (init) -- (s1);
%     \path [arrow1] (s1) -- (s2);
%     \path [arrow1] (s2) -- (s3);
%     \path [arrow1] (s3) -- node [right, near start] {Y} (s4);
%     \path [arrow1] (s3) -| node [above, near start] {N} ($(stop)+(3cm,0cm)$) -- (stop);
%     \path [arrow1] (s4) -- (stop);
    
% \end{tikzpicture}


%%%%%%%%%%%%%%%%%%%%%%%%%%%%%%% detail_optimizer21.pdf
% \begin{tikzpicture}
%     % Place nodes
%     \node [start] (init) {开始};
%     \node [process, below = of init] (s1) {优化表达式代码};
%     \node [decision, below = of s1] (s2) {表达式为常量?};
%     \node [process, below = of s2] (s3) {消除表达式代码};
%     \node [decision, below = of s3] (s4) {布尔值为 true ?};
%     \node [process, below = of s4] (s5) {优化 if 块,消除 else 块代码};
%     \node [process, right = of s5] (s6) {消除 if 块代码,优化 else 块代码};
%     \node [process, left = of s5] (s7) {优化 if 块代码,优化 else 块代码};
%     \node [start, below = of s5] (stop) {结束};

%     \path [arrow1] (init) -- (s1);
%     \path [arrow1] (s1) -- (s2);
%     \path [arrow1] (s2) -- node [right, near start] {Y} (s3);
%     \path [arrow1] (s2) -| node [above, near start] {N} (s7) |- (stop);
%     \path [arrow1] (s3) -- (s4);
%     \path [arrow1] (s4) -- node [right, near start] {Y} (s5);
%     \path [arrow1] (s4) -| node [above, near start] {N} (s6);
%     \path [arrow1] (s5) -- (stop);
%     \path [arrow1] (s6) |- (stop);
    
% \end{tikzpicture}


%%%%%%%%%%%%%%%%%%%%%%%%%%%%%%% detail_optimizer22.pdf
% \begin{tikzpicture}
%     % Place nodes
%     \node [start] (init) {开始};
%     \node [process, below = of init] (s1) {优化表达式代码};
%     \node [decision, below = of s1] (s2) {表达式为 false ?};
%     \node [process, below left = of s2] (s3) {消除while语句代码};
%     \node [process, below right = of s2] (s4) {优化while块代码};
%     \node [start, below = 2.3cm of s2] (stop) {结束};

%     \path [arrow1] (init) -- (s1);
%     \path [arrow1] (s1) -- (s2);
%     \path [arrow1] (s2) -| node [above, near start] {Y} (s3);
%     \path [arrow1] (s2) -| node [above, near start] {N} (s4);
%     \path [arrow1] (s3) |- (stop);
%     \path [arrow1] (s4) |- (stop);
    
% \end{tikzpicture}


%%%%%%%%%%%%%%%%%%%%%%%%%%%%%%% detail_optimizer23.pdf
\begin{tikzpicture}
    % Place nodes
    \node [start] (init) {开始};
    \node [process, below = of init] (s1) {优化表达式代码};
    \node [decision, below = of s1] (s2) {表达式为 0 ?};
    \node [process, below left = of s2] (s3) {消除repeat语句代码};
    \node [decision, below right = of s2] (s4) {表达式为 1 ?};
    \node [process, below left = of s4] (s5) {消除表达式代码、优化repeat块代码};
    \node [process, below right = of s4] (s6) {优化repeat块代码};
    \node [start, below = 3cm of s4] (stop) {结束};

    \path [arrow1] (init) -- (s1);
    \path [arrow1] (s1) -- (s2);
    \path [arrow1] (s2) -| node [above, near start] {Y} (s3);
    \path [arrow1] (s2) -| node [above, near start] {N} (s4);
    \path [arrow1] (s3) |- (stop);
    \path [arrow1] (s4) -| node [above, near start] {Y} (s5);
    \path [arrow1] (s4) -| node [above, near start] {N} (s6);
    \path [arrow1] (s5) |- (stop);
    \path [arrow1] (s6) |- (stop);
    
\end{tikzpicture}
\end{document}