\documentclass[border=5pt]{standalone}
%
\usepackage{basicsty}
%
\begin{document} 

%%%%%%%%%%%%%%%%%%%%%%%%%%%%%%%%%%% scanner.pdf
% \newcommand{\token}[2]{$\langle$ \texttt{\bfseries #1}, \texttt{#2} $\rangle$}
% \newcommand{\self}[1]{$\langle$ \texttt{\bfseries #1} $\rangle$}
% \begin{tikzpicture}
%     \node[rect1] (scanner) {Scanner};
%     \coordinate[above = of scanner] (tp);
%     \coordinate[below = of scanner] (bd);
%     % 绘制中间连线
%     \draw[arrow1]
%         (tp) node[above] {\verb!if [1 = a] { res : "ok"; }!}
%         -- node[right] {字符流} (scanner);
%     \draw[arrow1]
%         (scanner) 
%         -- node[right]{token流} 
%         (bd) node[below] {
%             \begin{minipage}{4cm}
%                 % \begin{center}
%                 \self{if}
%                 \self{[}
%                 \token{INT}{$1$}
%                 \self{=}

%                 \token{ID}{"a"}
%                 \self{]}
%                 $\langle \{\rangle$

%                 \token{ID}{"res"}
%                 \self{:}

%                 \token{STR}{"ok"}
%                 \self{;}
%                 $\langle \}\rangle$
%                 % \end{center}
%             \end{minipage}
%         };
% \end{tikzpicture} 


% %%%%%%%%%%%%%%%%%%%%%%%%%%%%%%%%%% parser1.pdf
% \newcommand{\token}[2]{$\langle$ \texttt{\bfseries #1}, \texttt{#2} $\rangle$}
% \newcommand{\self}[1]{$\langle$ \texttt{\bfseries #1} $\rangle$}
% \begin{tikzpicture}[level distance=1.5cm,
%     level 1/.style={sibling distance=2cm, level distance=1.3cm},
%     level 2/.style={sibling distance=1.3cm, level distance=1cm}]
%     \node [rect1] (parser) {Parser};
%     \coordinate [above = of parser] (tp);
%     \coordinate [below = of parser] (bd);
%     \node [below = of parser] {if\_statement}
%     child{node{$=$} child{node{1}} child{node{a}}}
%     child{node{block} child{node {$:$} child{node{res}} child{node{\texttt{"ok"}}}}}
%     ;
%     % 绘制中间连线
%     \draw[arrow1]
%         (tp) node[above] {
%             \begin{minipage}{4cm}
%                 % \begin{center}
%                 \self{if}
%                 \self{[}
%                 \token{INT}{$1$}
%                 \self{=}

%                 \token{ID}{"a"}
%                 \self{]}
%                 $\langle \{\rangle$

%                 \token{ID}{"res"}
%                 \self{:}

%                 \token{STR}{"ok"}
%                 \self{;}
%                 $\langle \}\rangle$
%                 % \end{center}
%             \end{minipage}
%         }
%         -- node[right] {token流} 
%         (parser);
%     \draw[arrow1]
%         (parser) 
%         -- node[right]{初步构建的语法树} 
%         (bd);
% \end{tikzpicture} 


%%%%%%%%%%%%%%%%%%%%%%%%%%%%%%%%%% parser2.pdf
% \newcommand{\token}[2]{$\langle$ \texttt{\bfseries #1}, \texttt{#2} $\rangle$}
% \newcommand{\self}[1]{$\langle$ \texttt{\bfseries #1} $\rangle$}
% \begin{tikzpicture}[level distance=1.5cm,
%     level 1/.style={sibling distance=4cm, level distance=1.3cm},
%     level 2/.style={sibling distance=1.5cm, level distance=1cm}]
%     \coordinate (tp);
%     \coordinate[below = of tp] (bd);
%     \node[below = of tp] {statement}
%     child {node{\self{if}}}
%     child {
%         node{expr}
%         child{node{\self{[}}}
%         child{
%             node{bool\_expr}
%             child{node{\token{INT}{1}}}
%             child{node{\self{=}}}
%             child{node{\token{ID}{a}}}
%         }
%         child{node{\self{]}}}
%     }
%     child {
%         node{block}
%         child{node{$\langle\{\rangle$}}
%         child{
%             node{statements}
%             child{node{略}}
%         }
%         child{node{$\langle\}\rangle$}}
%     };
%     \draw[arrow1]
%         (tp) node[above] {\verb!if [1 = a] { res : "ok"; }!}
%         -- 
%         (bd); % node[below] {1}child{node{2}}child{node{3}};
% \end{tikzpicture} 


%%%%%%%%%%%%%%%%%%%%%%%%%%%%%%% semantic.pdf
% \newcommand{\token}[1]{\texttt{\bfseries #1}}
% \begin{tikzpicture}[level distance=1.5cm,
%     level 1/.style={sibling distance=2cm, level distance=1.3cm},
%     level 2/.style={sibling distance=1.3cm, level distance=1cm}]
%     \node [rect1] (semantic) {语义分析子程序};
    
%     \coordinate [above = of semantic] (tp);
%     \node [above = 2.3cm of semantic] {$+$}
%     child{node{1.5}}
%     child{node{2}}
%     ;
%     \node [below = of semantic] (bd) {$+$}
%     child{node{1.5}}
%     child{node{\token{inttofloat}} child{node{2}}}
%     ;
%     % 绘制中间连线
%     \draw[arrow1]
%         (tp)
%         -- node[right] {初步构建的语法树} 
%         (semantic);
%     \draw[arrow1]
%         (semantic) 
%         -- node[right]{修正后的语法树} 
%         (bd);
% \end{tikzpicture} 


%%%%%%%%%%%%%%%%%%%%%%%%%%%%%%% semantic1.pdf
% \newcommand{\token}[1]{\texttt{\bfseries #1}}
% \begin{tikzpicture}[level distance=1.5cm,
%     level 1/.style={sibling distance=2cm, level distance=1.3cm},
%     level 2/.style={sibling distance=1.3cm, level distance=1cm}]
    
%     \coordinate (lp);
%     \coordinate [right = of lp] (rp);
%     \coordinate (lrt) at ($ (lp) + (-1.7cm, 1.3cm) $);
%     \coordinate (rrt) at ($ (rp) + (+1.7cm, 1.3cm) $);
%     \node at (lrt) {$>$}
%     child{node{$>$} child{node{3}} child{node{2}}}
%     child{node{1}}
%     ; 
%     \node at (rrt) {$\&\&$}
%     child{node{$>$} child{node{3}} child{node{2}}}
%     child{node{$>$} child{node{2}} child{node{1}}}
%     ;

%     \draw [arrow1] (lp) -- node[above]{修正} (rp);
% \end{tikzpicture} 


%%%%%%%%%%%%%%%%%%%%%%%%%%%%%%% semantic2.pdf
% \newcommand{\token}[1]{\texttt{\bfseries #1}}
% \begin{tikzpicture}[level distance=1.5cm,
%     level 1/.style={sibling distance=2cm, level distance=1.3cm},
%     level 2/.style={sibling distance=1.3cm, level distance=1cm}]
    
%     \coordinate (lp);
%     \coordinate [right = of lp] (rp);
%     \coordinate (lrt) at ($ (lp) + (-1.7cm, 0.65cm) $);
%     \coordinate (rrt) at ($ (rp) + (+2cm, 0cm) $);
%     \node at (lrt) {$+$}
%     child{node{1}}
%     child{node{\token{true}}}
%     ; 
%     \node at (rrt) {{\parbox[c]{3cm}{The operand type of the arithmetic ope-rator doesn't match.}}}
%     ;

%     \draw [arrow1] (lp) -- node[above]{报错} (rp);
% \end{tikzpicture} 


%%%%%%%%%%%%%%%%%%%%%%%%%%%%%%% generator.pdf
% \newcommand{\token}[1]{\texttt{\bfseries #1}}
% \begin{tikzpicture}[level distance=1.5cm,
%     level 1/.style={sibling distance=2cm, level distance=1.3cm},
%     level 2/.style={sibling distance=1.3cm, level distance=1cm}]
%     \node [rect1] (generator) {代码生成子程序};
    
%     \coordinate [above = of generator] (tp);
%     \node [above = 3.3cm of generator] {$+$}
%     child{node{1.5}}
%     child{node{\token{inttofloat}} child{node{2}}}
%     ;
%     \node [below = of generator] (bd) {
%         \begin{minipage}{4cm}
% \begin{lstlisting}[numbers=none]
% 0000: push_float         0000
% 0003: push_byte          02
% 0005: cast_int_to_float 
% 0006: add_float         
% \end{lstlisting}
%         \end{minipage}
%     }
%     ;
%     % 绘制中间连线
%     \draw[arrow1]
%         (tp)
%         -- node[right] {修正后的语法树} 
%         (generator);
%     \draw[arrow1]
%         (generator) 
%         -- node[right]{字节码} 
%         (bd);
% \end{tikzpicture} 


%%%%%%%%%%%%%%%%%%%%%%%%%%%%%%% generator1.pdf
% \newcommand{\token}[1]{\texttt{\bfseries #1}}
% \begin{tikzpicture}[level distance=1.5cm,
%     level 1/.style={sibling distance=2cm, level distance=1.3cm},
%     level 2/.style={sibling distance=1.3cm, level distance=1cm}]
    
%     \coordinate (lp);
%     \coordinate [right = of lp] (rp);
%     \coordinate (lrt) at ($ (lp) + (-2cm, 1.3cm) $);
%     \coordinate (rrt) at ($ (rp) + (+2.7cm, 0cm) $);
%     \node at (lrt) {if\_statement}
%     child{node{$=$} child{node{1}} child{node{a}}}
%     child{node{block} child{node {$:$} child{node{res}} child{node{\texttt{"ok"}}}}}
%     ;
%     \node at (rrt) {
%         \begin{minipage}{4cm}
% \begin{lstlisting}[numbers=none]
% 0000: push_byte          01
% 0002: push_static_int    0000
% 0005: eq_int            
% 0006: jump_if_false      0012
% 0009: push_str           0000
% 000C: pop_static_object  0001
% 000F: jump               0012
% \end{lstlisting}
%         \end{minipage}
%     }
%     ;

%     \draw [arrow1] (lp) -- (rp);
% \end{tikzpicture} 


%%%%%%%%%%%%%%%%%%%%%%%%%%%%%%% optimizer.pdf
% \newcommand{\token}[1]{\texttt{\bfseries #1}}
% \begin{tikzpicture}[level distance=1.5cm,
%     level 1/.style={sibling distance=2cm, level distance=1.3cm},
%     level 2/.style={sibling distance=1.3cm, level distance=1cm}]
%     \node [rect1] (optimizer) {代码优化子程序};
    
%     \coordinate [above = of optimizer] (tp);
%     \node [above = of optimizer] (bd) {
%         \begin{minipage}{4cm}
% \begin{lstlisting}[numbers=none]
% 0000: push_float         0000
% 0003: push_byte          02
% 0005: cast_int_to_float 
% 0006: add_float         
% \end{lstlisting}
% \vspace{-.2cm}
%         \end{minipage}
%     }
%     ;
%     \node [below = of optimizer] (bd) {
%         \vspace{.3cm}
%         \begin{minipage}{4cm}
% \begin{lstlisting}[numbers=none]
% 0000: push_float         0000   
% \end{lstlisting}
%         \end{minipage}
%     }
%     ;
%     % 绘制中间连线
%     \draw[arrow1]
%         (tp)
%         -- node[right] {字节码} 
%         (optimizer);
%     \draw[arrow1]
%         (optimizer) 
%         -- node[right]{优化后的字节码} 
%         (bd);
% \end{tikzpicture} 


%%%%%%%%%%%%%%%%%%%%%%%%%%%%%%% optimizer1.pdf
% \newcommand{\token}[1]{\texttt{\bfseries #1}}
% \begin{tikzpicture}[level distance=1.5cm,
%     level 1/.style={sibling distance=2cm, level distance=1.3cm},
%     level 2/.style={sibling distance=1.3cm, level distance=1cm}]
    
%     \coordinate (lp);
%     \coordinate [right = 2.3cm of lp] (rp);
%     \coordinate (lrt) at ($ (lp) + (-2cm, 1.3cm) $);
%     \coordinate (rrt) at ($ (rp) + (+2cm, 1.3cm) $);
%     \node at (lrt) {$-$}
%     child{node{$+$} child{node{1}} child{node{$*$} child{node{2}} child{node{3}}}}
%     child{node{t}}
%     ;
%     \node at (rrt) {$-$}
%     child{node{7}}
%     child{node{t}}
%     ;

%     \draw [arrow1] (lp) -- node [above] {常量折叠} (rp);
% \end{tikzpicture} 


%%%%%%%%%%%%%%%%%%%%%%%%%%%%%%% optimizer2.pdf
% \newcommand{\token}[1]{\texttt{\bfseries #1}}
% \begin{tikzpicture}[level distance=1.5cm,
%     level 1/.style={sibling distance=2cm, level distance=1.3cm},
%     level 2/.style={sibling distance=1.3cm, level distance=1cm}]
    
%     \coordinate (lp);
%     \coordinate [right = 2.3cm of lp] (rp);
%     \coordinate (lrt) at ($ (lp) + (-3.5cm, 1.3cm) $);
%     \coordinate (rrt) at ($ (rp) + (+2.5cm, 1.3cm) $);
%     \node at (lrt) {\token{block}}
%     child{node{\token{println}} child{node{\texttt{"before return"}}}}
%     child{node{\token{return}} child{node{0}}}
%     child{node{\token{println}} child{node{\texttt{"after return"}}}}
%     ;
%     \node at (rrt) {\token{block}}
%     child{node{\token{println}} child{node{\texttt{"before return"}}}}
%     child{node{\token{return}} child{node{0}}}
%     ;

%     \draw [arrow1] (lp) -- node [above] {死码消除} (rp);
% \end{tikzpicture} 


%%%%%%%%%%%%%%%%%%%%%%%%%%%%%%% optimizer3.pdf
% \newcommand{\token}[1]{\texttt{\bfseries #1}}
% \begin{tikzpicture}[level distance=1.5cm,
%     level 1/.style={sibling distance=2cm, level distance=1.3cm},
%     level 2/.style={sibling distance=2cm, level distance=1cm},
%     level 3/.style={sibling distance=1.3cm, level distance=1cm}]
    
%     \coordinate (lp);
%     \coordinate [right = 2.3cm of lp] (rp);
%     \coordinate (lrt) at ($ (lp) + (-2.5cm, 1.3cm) $);
%     \coordinate (rrt) at ($ (rp) + (+2cm, 1.3cm) $);
%     \node at (lrt) {\token{outer\_block}}
%     child {
%         node{\token{repeat}}
%         child{node{$-$} child{node{3}} child{node{2}}}
%         child{node{\token{inner\_block}} child{node{略}}}
%     }
%     ;
%     \node at (rrt) {\token{outer\_block}}
%     child {node{\token{inner\_block}} child{node{略}}}
%     ;

%     \draw [arrow1] (lp) -- node [above] {常量折叠} node [below] {repeat 展开} (rp);
% \end{tikzpicture} 


%%%%%%%%%%%%%%%%%%%%%%%%%%%%%%% flex1.pdf
% \begin{minipage}{10cm}
% \begin{lstlisting}[numbers=none,language=C++]
% [A-Za-z_][A-Za-z_0-9]*  {
%     return make_IDENTIFIER(std::string(YYText()));
% }
% \end{lstlisting}
% \end{minipage}


%%%%%%%%%%%%%%%%%%%%%%%%%%%%%%% flex2.pdf
% \begin{minipage}{10cm}
% \begin{lstlisting}[numbers=none,language=C++]
% <INITIAL>"(*"     depthOfComment = 1, BEGIN BO_COMMENT;
% <BO_COMMENT>{
% \n      increment_line_number();
% "(*"    ++depthOfComment;
% "*)"    if ( --depthOfComment == 0 ) BEGIN INITIAL;
% <<EOF>> throw syntax_error("Unexpected end of line in comment.");
% .       ;  // 空语句
% }
% \end{lstlisting}
% \end{minipage}


%%%%%%%%%%%%%%%%%%%%%%%%%%%%%%% bison1.pdf
% \begin{minipage}{10cm}
% \begin{lstlisting}[numbers=none,language=C++]
% repeat_statement : REPEAT expression block {
%     $$ = create_repeat_statement($expression, $block);
% };
% \end{lstlisting}
% \end{minipage}


%%%%%%%%%%%%%%%%%%%%%%%%%%%%%%% characteristic.pdf
% \begin{minipage}{4.5cm}
% \begin{lstlisting}[numbers=none,language=C++]
% (*
% define say_hello() -> void {
%     println("hello");
%     (* println("world"); *)
% }

% say_hello();
% *)
% print("ok");  # ok
% \end{lstlisting}
% \end{minipage}


%%%%%%%%%%%%%%%%%%%%%%%%%%%%%%% flowChart.pdf
% \begin{tikzpicture}
%     % Place nodes
%     \node [start] (init) {开始};
%     \node [process, below = of init] (step1) {对当前文件进行词法和语法分析};
%     \node [decision, below = of step1] (decide) {需要导包?};
%     \node [process, below = of decide] (step2) {对当前文件进行语义分析、代码生成和代码优化};
%     \node [process, below = of step2] (step3) {为当前文件生成 \texttt{.boc} 文件};
%     \node [start, below = of step3] (stop) {结束};
%     \node at ($(decide) + (4cm, -1.4cm)$) [process] (load) {依次编译所需外部文件};
%     % Draw edges
%     \path [arrow1] (init) -- (step1);
%     \path [arrow1] (step1) -- (decide);
%     \path [arrow1] (decide) -| node [above, near start] {yes} (load);
%     \path [arrow1] (load) |- (step2);
%     \path [arrow1] (decide) -- node [right] {no} (step2);
%     \path [arrow1] (step2) -- (step3);
%     \path [arrow1] (step3) -- (stop);
% \end{tikzpicture}


\end{document}



% \begin{tikzpicture}
%     \node[rect1] (scanner) {Scanner};
%     \node[rect1, below = of scanner] (parser) {Parser};
%     \node[rect1, below = of parser] (semantic) {语义分析子程序};
%     \node[rect1, below = of semantic] (generator) {代码生成子程序};
%     \node[rect1, below = of generator] (optimizer) {代码优化子程序};
%     \coordinate[above = of scanner] (tp);
%     % 绘制中间连线
%     \draw[arrow1](tp)  -- node[right] {字符流} (scanner);
%     \draw[arrow1](scanner) -- node[right]{token流} (parser);
%     \draw[arrow1](parser) -- node[right]{初步构建的语法树} (semantic);
%     \draw[arrow1](semantic) -- node[right]{修正后的语法树} (generator);
%     \draw[arrow1](generator) -- node[right]{字节码} (optimizer);
%     \draw[arrow1](optimizer) -- node[right]{优化后的字节码}++(0, -50pt);
    
%     % \coordinate (p1) at ($ (compiler) + (1.65cm, 0.5cm) $);
%     % \coordinate (p2) at ($ (scanner) + (-1.65cm, 0.5cm) $);
%     % \coordinate (p3) at ($ (compiler) + (1.65cm, -0.5cm) $);
%     % \coordinate (p4) at ($ (optimizer) + (-1.65cm, -0.5cm) $);
%     % \draw[dashed] (p1) -- (p2);
%     % \draw[dashed] (p3) -- (p4);
% \end{tikzpicture} 