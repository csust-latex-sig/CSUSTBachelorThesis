\documentclass[border={0pt 15pt 70pt 15pt}]{standalone}
%
\usepackage{basicsty}
%
\begin{document} 

\begin{tikzpicture}
\node (hello_bo) {
\begin{minipage}{3.3cm}
\begin{center}
    hello.bo
\end{center}
\vspace{-.9\baselineskip}

\begin{lstlisting}[numbers=none,linewidth=3.3cm]
# hello.bo
print("Hello world!");
\end{lstlisting}
\vspace{-.6\baselineskip}

\end{minipage}
};
\node[rect1, below = of hello_bo] (compiler) {BO编译器};
\node[below = of compiler] (hello_boc) {
\begin{minipage}{3.3cm}
\begin{center}
    hello.boc
\end{center}
\vspace{-.9\baselineskip}

\begin{lstlisting}[numbers=none,linewidth=3.3cm]
4DE6 40BB 2605 2220 ...
\end{lstlisting}
\vspace{-.6\baselineskip}

\end{minipage}
};
\node[rect1, below = of hello_boc] (bovm) {BOVM};
\node[below = of bovm] (output) {
\begin{minipage}{3.3cm}
终端打印:
\vspace{-.2\baselineskip}

\begin{lstlisting}[numbers=none,linewidth=3.3cm]
Hello world!
\end{lstlisting}
\end{minipage}
};
\node[left = 2cm of compiler] (lang_bo) {};
\node[left = 2cm of bovm] (lang_boc) {};
\draw[arrow1] (hello_bo) -- node[right]{boc命令} (compiler);
\draw[arrow1] (compiler) -- (hello_boc);
\draw[arrow1] (hello_boc) -- node[right]{bo命令} (bovm);
\draw[arrow1] (bovm) -- (output);
\draw[arrow1] (lang_bo) -- node[above]{lang.bo} (compiler);
\draw[arrow1] (lang_boc) -- node[above]{lang.boc} (bovm);
% \draw (0,0) node[above right] {hello.c} -- (1,1);
% \end{tikzpicture}

% \begin{tikzpicture}
    % 绘制中间方框
    \node[rect1, right = 3cm of hello_bo] (scanner) {Scanner};
    \node[rect1, below = of scanner] (parser) {Parser};
    \node[rect1, below = of parser] (semantic) {语义分析子程序};
    \node[rect1, below = of semantic] (generator) {代码生成子程序};
    \node[rect1, below = of generator] (optimizer) {代码优化子程序};
    \coordinate[above = of scanner] (tp);
    % 绘制中间连线
    \draw[arrow1](tp)  -- node[right] {字符流} (scanner);
    \draw[arrow1](scanner) -- node[right]{token流} (parser);
    \draw[arrow1](parser) -- node[right]{初步构建的语法树} (semantic);
    \draw[arrow1](semantic) -- node[right]{修正后的语法树} (generator);
    \draw[arrow1](generator) -- node[right]{字节码} (optimizer);
    \draw[arrow1](optimizer) -- node[right]{优化后的字节码}++(0, -50pt);
    % \draw[arrow1](compiler) .. controls (tmp_point) .. (scanner);
    % 绘制虚线框
    % \draw[thick, dashed, draw = purple](-125pt, 27pt)node[below right]{编译器前端} -- (125pt, 27pt) -- (125pt, -250pt) -- (-125pt, -250pt)node[above right]{编译器后端} -- (-125pt, 27pt);
    % \draw[thick, dashed, draw = purple](-125pt, -135pt) -- (125pt, -135pt);
    \coordinate (p1) at ($ (compiler) + (1.65cm, 0.5cm) $);
    \coordinate (p2) at ($ (scanner) + (-1.65cm, 0.5cm) $);
    \coordinate (p3) at ($ (compiler) + (1.65cm, -0.5cm) $);
    \coordinate (p4) at ($ (optimizer) + (-1.65cm, -0.5cm) $);
    \draw[dashed] (p1) -- (p2);
    \draw[dashed] (p3) -- (p4);
 \end{tikzpicture} 

\end{document}