\chap{绪论}

\sect{研究背景}
长沙理工大学教务处官网上的\href{https://www.csust.edu.cn/jwc/sjjx/bysj.htm}{毕业设计word模板}\footnote{本文撰写时(2022年6月),官网的毕设模板发布时间为2020年10月。若今后官网更新了模板,本节内容仅供参考。}存在许多值得诟病的地方,比如:封面和扉页存在中英文冒号混用的问题;任务书封面存在有些空无法直接填写、整体内容未居中的问题;此外,封面、扉页、诚信声明和任务书均存在页边距不符合规范的问题。

上述\href{https://www.csust.edu.cn/jwc/sjjx/bysj.htm}{毕业设计word模板}问题虽小,但对苦于论文内容撰写的长理学子来说真的是雪上加霜,将多个word的内容合并时还要经历新一轮的折磨。


\sect{国内外研究现状}
麻省理工、清北复交等众多国际国内知名大学早已拥有自己的 \LaTeX{} 模板,而长沙理工大学至今(2022年6月)无人发起与维护一份像样的论文模板。此前在 github 上查询 “csust thesis” 关键词,仅出现两个结果,一个2016年5月发布的 \href{https://github.com/zonkisa/LaTeX-thesis-model-for-CSUST}{LaTeX-thesis-model-for-CSUST} ,然而这是个空项目;另一个是2019年3月发布的\href{https://github.com/leoppro/csust_thesis_latex}{长沙理工大学学士学位论文 LateX 模板},其中仅含开题报告和任务书,而且没有任何说明性文档。

\sect{研究意义}
该研究会使长沙理工大学也拥有一份非官方的 \LaTeX{} 模板 ,可以为今后想使用 \LaTeX 撰写毕业论文的学弟学妹带来方便。

\sect{论文结构}
本文共包含五个章节。

\vspace{-1.5\baselineskip}
\begin{description}
    \item[\hspace{2em}]\!\!
\begin{description}
    \item[第一章] 绪论。简述本课题的研究背景、简述国内外研究现状、阐明课题研究内容及意义。
    \item[第二章] 关于\LaTeX{}。简要介绍 \LaTeX{} 是什么、\TeX{} Live 发行版的安装以及编辑器的选择和配置。
    \item[第三章] 模板设计与实现。介绍本模板的设计细节。
    \item[第四章] 模板使用说明。介绍如何使用本模板。
    \item[第五章] 总结与展望。总结本模板当前所作工作与不足,展望本模板的未来发展。
\end{description} 
\end{description}