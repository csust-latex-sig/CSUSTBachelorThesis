\chap{绪论}

\sect{研究背景}
长沙理工大学教务处官网上的\href{https://www.csust.edu.cn/jwc/sjjx/bysj.htm}{毕业设计word模板}\footnote{本文撰写时(2022年6月),官网的毕设模板发布时间为2020年10月。若今后官网更新了模板,本节仅供参考。}存在许多值得诟病的地方。


\sect{国内外研究现状}
清华北大、复旦上交等众多国内知名大学早已拥有自己的 \LaTeX 模板,而长沙理工大学至今(2022年6月)无人发起与维护。

\sect{研究意义}
该研究会使长沙理工大学也拥有一份非官方的 \LaTeX ,可以为今后想使用 \LaTeX 撰写毕业论文的学弟学妹带来方便。

\sect{论文结构}

\vspace{-1.5\baselineskip}
\begin{description}
    \item[\hspace{2em}]\!\!
\begin{description}
    \item[第一章] 绪论。简述本课题的研究背景、简述国内外研究现状、阐明课题研究内容及意义。
    \item[第二章] 需求分析。
    \item[第三章] 软件工具原理与优缺点。介绍设计和实现BO编译器所用到的软件工具及其原理,包括词法分析器和语法分析器的自动生成工具 Flex 和 Bison 、用于描述上下文无关文法的 BNF 范式、以及跨平台的项目打包工具 Cmake 等。
    \item[第四章] 总体设计。设计 BO 编译器各模块的接口及功能,介绍 BO 语言的特性及运行过程,简述 BOVM 虚拟机的工作方式。
    \item[第五章] 详细设计与系统实现。详细介绍 BO 编译器各阶段基本功能的实现以及 BO 编译器对部分 BO 语言特性的实现。
    \item[第六章] 测试。通过对BO编译器进行系统化测试找出编译器存在的潜在问题、检验编译器的质量。
\end{description} 
\end{description}