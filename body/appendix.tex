\appendix  % 附录开始

\chap{附录撰写示例}  % 附录 A
\sect{附录的二级标题实例}
本模板不会将附录的二级标题加入目录,如有需要,请自行实现。
\subsect{附录的三级标题实例}
本模板不会将附录的三级标题加入目录,如有需要,请自行实现。
\sect{附录图表示例}

长沙理工大学校徽如图 \ref{fig:appendix:csustlogo} 所示。  % 使用 \ref 命令引用图片标签

\begin{figure}[htbp]  % htbp选项允许浮动体出现在任意位置,后文详述
  \centering  % 表格居中
  \includegraphics[scale=.5]{figure/template/csustlogo_626by572.jpg}  % 选项代表缩小到原图的 0.5 倍。
  \caption{长沙理工大学校徽}  % 图题
  \label{fig:appendix:csustlogo}  % 定义标签,方便引用该图
\end{figure}

浮动体选项及其含义如表 \ref{tab:appendix:floatchoice} 所示。

\begin{table}[htbp]
  \centering  % 表格居中
  \zihao{5}  % 五号字体
  \caption{浮动体选项及其含义}
  % \vspace{-3mm}
  \begin{tabular}[]{ll}
      \toprule   % 三线表的第一条线
      表项      & 含义            \\
      \midrule   % 三线表的第二条线
        !       &  忽略一些严格的限制   \\
        h       &  如果可以,放在当前位置   \\
        t       &  允许放在顶部   \\
        b       &  允许放在底部  \\
        p       &  运行放在浮动栏或浮动页   \\
        H       &  禁止浮动   \\
      \bottomrule   % 三线表的第三条线
  \end{tabular}
  \label{tab:appendix:floatchoice}  % 注意:这不是伪代码
\end{table}


\chap{文档类 csustThesis 完整代码}  % 附录 B
\begin{lstlisting}[numbers=none,frame=none]
%%%%%%%%%%%%%%%%%%%%%%%%%%%%%%%%%%%%%%%%%%%%%%%%%%%%%%%%%%%%
% 文档类 csustThesis 的完整代码以文件 csustThesis.cls 中的内容为准。
%%%%%%%%%%%%%%%%%%%%%%%%%%%%%%%%%%%%%%%%%%%%%%%%%%%%%%%%%%%%

% Identification
\NeedsTeXFormat{LaTeX2e}
\ProvidesClass{csustThesis}[2022/05/11 Unofficial LaTeX Document Class for CSUST Bachelor Thesis]

% Preliminary declarations
%%%%%%%%%%%%%%%%%%%%%%%%%%%%%%%%%%%%%%%%%%%%%%%%%% 基本文档类及引入的宏包 %%%%%%%%%%%%%%%%%%%%%%%%%%%%%%%%%%%%%%%%%%%%%%%%%%
\LoadClass{article}
\RequirePackage[zihao=-4,heading=true,linespread=1.5]{ctex}
\RequirePackage[a4paper,top=25mm,bottom=20mm,left=30mm,right=20mm]{geometry}
\RequirePackage[colorlinks, linkcolor=blue, anchorcolor=yellow, citecolor=red]{hyperref}
\usepackage{amssymb, amsfonts, amsmath, amsthm, mathtools, mathrsfs, bm, calc} %% 和数学相关的一些宏包
\RequirePackage{bookmark}
\RequirePackage{graphicx}
\RequirePackage{setspace}
\RequirePackage{fancyhdr}
\RequirePackage{needspace}
\RequirePackage{lastpage}
\RequirePackage{chngcntr}  % 图标按章节编号
\RequirePackage{placeins}  % 图片仅限本章浮动
\RequirePackage{float}  % 浮动体的 H 选项
\RequirePackage{caption,subcaption}  % 标题控制
\RequirePackage{listings}  % 代码环境
\RequirePackage{xcolor}  % 颜色支持
\RequirePackage{tcolorbox}  % 文本框
\RequirePackage{longtable}  % 长表格
\RequirePackage{makecell}  % 表格线宽支持
\RequirePackage{booktabs}  % 三线表支持
\RequirePackage{bigstrut} %% 该宏包提供 \bigstrut 命令,可以增大表格的行间距
\RequirePackage[backend=biber,style=gb7714-2015,gbnamefmt=lowercase,refsegment=part,defernumbers=true,gbtype=true,]{biblatex}  % 支持参考文献
\RequirePackage{emptypage} %% 此宏包负责的任务是当每一章最后一页是偶数页时,设置空白
\RequirePackage{tikz}
\RequirePackage{indentfirst}
\RequirePackage[final]{pdfpages}
\RequirePackage{titlesec}  % titlespacing
\RequirePackage[titles]{tocloft}  % 用于设置目录条目
\RequirePackage{suffix}  % 定义 * 命令
\RequirePackage{varwidth}  
\RequirePackage{ulem}  % 删除线  

% Options

% More declarations

%%%%%%%%%%%%%%%%%%%%%%%%%%%%%%%%%%%%%%%%%%%%%%%%%% 全局设置 %%%%%%%%%%%%%%%%%%%%%%%%%%%%%%%%%%%%%%%%%%%%%%%%%%
%====================添加中文字体族====================%
\setCJKfamilyfont{SimSun}[AutoFakeBold={2.17}]{宋体}
\setCJKfamilyfont{SimHei}[AutoFakeBold={2.17}]{黑体}
\setCJKfamilyfont{KaiTi}[AutoFakeBold={2.17}]{楷体}
%====================定义字体命令====================%
\def\songti{\CJKfamily{SimSun}}
\def\heiti{\CJKfamily{SimHei}}
\def\kaiti{\CJKfamily{KaiTi}}
% 设置默认字体,中文默认宋体,西文默认 Times New Roman
\setCJKmainfont{宋体}[AutoFakeBold={2.17}]  % 中文默认字体
\setmainfont{Times New Roman}  % 西文默认字体

% 章节标题
\ctexset{
	section = {  % 一级标题
		format = \centering\heiti\zihao{-3},  % 黑体小三居中
		name = {第, 章},
	},
	subsection = {  % 二级标题
		format = \heiti\zihao{4},  % 黑体四号
	},
	subsubsection = {  % 三级标题
		format = \heiti\zihao{-4},  % 黑体小四
        indent = 2\ccwd,  % 缩进两字符
	}
}

% 定义标题命令
\newcommand{\chap}[1]{%
	\clearpage  % 每章另起一页
    \FloatBarrier  % 清除浮动体,也就是输出上一章的图表
    \vspace*{-.5\baselineskip}  % 调整章标题与页眉顶部的间距
	\section{#1}
}
\newcommand{\sect}[1]{\subsection{#1}}
\newcommand{\subsect}[1]{\subsubsection{#1}}

% 附录之后的章节标题及图表设置
\let\old@appendix\appendix
\renewcommand\appendix{
    \old@appendix
    \addcontentsline{toc}{section}{附录}  % 添加附录到目录
    \ctexset{section/format=\centering\heiti\zihao{3}}  % 黑体三号居中
    \def\chap##1{%
        \clearpage  % 每章另起一页
        \FloatBarrier  % 清除浮动体,也就是输出上一章的图表
        \vspace*{-.5\baselineskip}  % 调整章标题与页眉顶部的间距
        \stepcounter{section}
        \section*{附录\thesection\hspace{\ccwd}##1}
        \addcontentsline{toc}{subsection}{附录\thesection\hspace{\ccwd}##1}  % 为一级标题添加目录项
    }
    \def\sect##1{
        \setcounter{subsubsection}{0}
        \stepcounter{subsection}
        \subsection*{\thesubsection\hspace{\ccwd}##1}
        % \addcontentsline{toc}{subsubsection}{\thesubsection\hspace{\ccwd}##1}  % 为二级标题添加目录项
    }
    \def\subsect##1{
        \stepcounter{subsubsection}
        \subsubsection*{\thesubsubsection\hspace{\ccwd}##1}
    }
    \renewcommand{\thefigure}{\thesection\arabic{figure}}  % 图几几
    \renewcommand{\thetable}{\thesection\arabic{table}}  % 表几几
}

% 页眉页脚预设置
\pagestyle{fancy}
\fancyhf{}

% 页眉设置
\fancyhead[L]{  % 设置页眉左侧
    \raisebox{-.1cm}{\includegraphics[height=1.15cm]{figure/csust_logo_and_name2.pdf}}
}
\fancyhead[R]{\let\\=\relax\heiti\zihao{-5}\csustThesis@title}  % 设置页眉右侧

% 长度变量,定义各种封面时使用
\newlength{\maxwidth}
\newcommand{\titleBoxLen}{11cm}
\newcommand{\len}{8cm}
\newcommand{\llen}{6cm}

% 图表名格式设置
\captionsetup{font=small,labelsep=space}  % 图表标题设为五号字体、序号与图表名之间以空格分隔
\captionsetup[table]{font={small,bf},labelsep=space}  % 表标题设为五号加粗、表序与表名之间以空格分隔
\counterwithin{figure}{section}  % 图按章节编号
\counterwithin{table}{section}  % 表按章节编号
\renewcommand{\thefigure}{\thesection.\arabic{figure}}  % 图几.几
\renewcommand{\thetable}{\thesection-\arabic{table}}  % 表几-几

% 增大表格间距
% \renewcommand{\arraystretch}{1.2}

% 代码抄录的一系列设置
\lstset{
    basicstyle=\footnotesize\ttfamily, 
    flexiblecolumns, 
    numbers=left, 
    numbersep=5pt, 
    numberstyle=\tiny, 
    backgroundcolor=\color{white}, 
    frame=single, 
    breaklines=true, 
    showtabs=false, 
    showspaces=false, 
    showstringspaces=false, 
    keywordstyle=\color{teal}, 
    commentstyle=\color{blue}, 
    stringstyle=\color{red}, 
    numberstyle=\color{gray}, 
    tabsize=8, 
    breakatwhitespace=false, 
    postbreak=\mbox{\textcolor{violet}{$\hookrightarrow$}\space}
} %% 代码抄录的一系列设置

%%%%%%%%%%%%%%%%%%%%%%%%%%%%%%%%%%%%%%%%%%%%%%%%%%%%%%%%%%%%
% 文档类 csustThesis 的完整代码参见 csustThesis.cls 文件。
%%%%%%%%%%%%%%%%%%%%%%%%%%%%%%%%%%%%%%%%%%%%%%%%%%%%%%%%%%%%
\end{lstlisting}
% 文档类 csustThesis 完整代码参见 csustThesis.cls 文件。


\chap{插图与表格汇总}  % 附录 C

\listoffigures

\listoftables
