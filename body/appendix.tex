\appendix  % 附录开始

\chap{附录撰写示例}  % 附录 A
\sect{附录的二级标题实例}
本模板不会将附录的二级标题加入目录,如有需要,请自行实现。
\subsect{附录的三级标题实例}
本模板不会将附录的三级标题加入目录,如有需要,请自行实现。
\sect{附录图表示例}

长沙理工大学校徽如图 \ref{fig:appendix:csustlogo} 所示。  % 使用 \ref 命令引用图片标签

\begin{figure}[htbp]  % htbp选项允许浮动体出现在任意位置,后文详述
  \centering  % 表格居中
  \includegraphics[scale=.5]{figure/csustlogo_626by572.jpg}  % 选项代表缩小到原图的 0.5 倍。
  \caption{长沙理工大学校徽}  % 图题
  \label{fig:appendix:csustlogo}  % 定义标签,方便引用该图
\end{figure}

浮动体选项及其含义如表 \ref{tab:appendix:floatchoice} 所示。

\begin{table}[htbp]
  \centering  % 表格居中
  \zihao{5}  % 五号字体
  \caption{浮动体选项及其含义}
  % \vspace{-3mm}
  \begin{tabular}[]{ll}
      \toprule   % 三线表的第一条线
      表项      & 含义            \\
      \midrule   % 三线表的第二条线
        !       &  忽略一些严格的限制   \\
        h       &  如果可以,放在当前位置   \\
        t       &  允许放在顶部   \\
        b       &  允许放在底部  \\
        p       &  运行放在浮动栏或浮动页   \\
        H       &  禁止浮动   \\
      \bottomrule   % 三线表的第三条线
  \end{tabular}
  \label{tab:appendix:floatchoice}  % 注意:这不是伪代码
\end{table}


\chap{文档类 csustThesis 完整代码}  % 附录 B
\sect{标题1}
\subsect{标题11}
\sect{标题2}
\subsect{标题21}


\chap{插图与表格汇总}  % 附录 C

\listoffigures

\listoftables