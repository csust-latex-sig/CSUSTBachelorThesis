\chap{总结与展望}

\sect{总结}
至今(2022年6月),基于 \LaTeX{} 系统的长沙理工大学本科生毕业论文模板已经实现了 \href{https://www.csust.edu.cn/jwc/info/1142/3568.htm}{《长沙理工大学本科毕业设计撰写规范》} 中的大部分要求,如页面格式、字体字号、图表样式和附录要求等。

但是本模板依然存在许多不足,比如:
\begin{itemize}
  \item 未支持跨页表格
  \item 未考虑公式自动编号
  \item 开题报告封面的课题类别默认选择“设计”,未提供更改接口。
  \item 参考文献样式采用 GB7714-2015 标准,与撰写规范要求的 GB7714-87 标准略有不同,需要在使用时加以注意。
  \item 任务书最后三面尚未制作。
  \item 文档有待完善。
  \item 在 Windows 系统上开发,未在其他平台上测试。
  \item 更多不足有待发现……
\end{itemize}

\sect{长理 \LaTeX{} 模板的未来}
该模板本身还存在很多不足之处、长沙理工大学的论文撰写规范在今后也可能会发生变化,因此这份模板需要一届又一届长理学子接力维护和发展。

希望长理 \LaTeX{} 模板越来越好!